\centergraphics[width=7cm]{p-082-tone4.pdf}

\hKv{Во ст҃ꙋ́ю и҆ вели́кꙋю четыредесѧ́тницꙋ стїхосло́вѧтсѧ  пѣ̑сни си́це.} 
\label{kakofty}

\hKv{По возгла́сѣ сщ҃е́нника,}
Млⷭ҇тїю и҆ щедро́тами:
\hKv{начина́емъ:}
Гдⷭ҇еви пое́мъ: 
\hKv{на гла́съ канѡ́на мине́и днѐ ст҃а́гѡ, и҆  глаго́лютъ стїхѝ пѣ́сней ско́рѡ, кі́йждо ли́къ сво́й  сті́хъ, до́ндеже дости́гнꙋтъ до}
Ѡ҆гꙋстѣ́ша:
\hKv{и҆ ѿ тогѡ̀  начина́ютъ стїхѝ держа́ти на д҃і. И҆ пое́мъ мине́и, со  і҆рмосо́мъ на ѕ҃: пое́мъ же си́це. Пе́рвый ли́къ глаго́летъ  сті́хъ:}
Ѡ҆гꙋстѣ́ша:
\hKv{и҆ пое́тъ і҆рмо́съ. Вторы́й же ли́къ  глаго́летъ вторы́й сті́хъ:}
Речѐ вра́гъ: 
\hKv{и҆ по не́мъ  тропа́рь канѡ́на. И҆ порѧ́дꙋ стїхѝ, по ликѡ́мъ, до}
Тогда̀  потща́шасѧ:
\hKv{Ѿ стїха́ же,}
Тогда̀ потща́шасѧ:
\hKv{начина́емъ  трипѣ́снецъ господи́на і҆ѡ́сифа, на д҃. На}
Гдⷭ҇ь \hKv{же}  црⷭ҇твꙋѧй: 
\hKv{пое́мъ дрꙋгі́й трипѣ́снецъ, господи́на ѳео́дѡра   стꙋді́та. И҆ пое́тъ є҆ди́нъ  ли́къ тропа́рь. Дрꙋгі́й же ли́къ вторы́й сті́хъ, и҆  тропа́рь. 
Посе́мъ соедини́вшымсѧ двꙋ́мъ ликѡ́мъ, пою́тъ,  Сла́ва: трⷪ҇ченъ, вы́шшимъ гла́сомъ: И҆ ны́нѣ:  бг҃оро́диченъ. Та́же глаго́лемъ высоча́йшимъ гла́сомъ:}

\hKv Сла́ва тебѣ̀ бж҃е на́шъ, сла́ва тебѣ̀. 
\hKv{И҆ пое́мъ и҆ны́й  тропа́рь господи́на ѳео́дѡра. Поне́же бо по пѧтѝ тропаре́й  и҆́мать пѣ́снь. Послѣди́ же глаго́лемъ катава́сїю,  
і҆рмо́съ втора́гѡ канѡ́на. Си́це пое́тсѧ и҃_ѧ и҆ ѳ҃_ѧ  пѣ̑сни. Тре́тїю же пѣ́снь пое́мъ си́це. Глаго́летъ пе́рвый  ли́къ сті́хъ:}
Гдⷭ҇ь взы́де на нб҃са̀:
\hKv{и тропа́рь канѡ́на  мине́и. Вторы́й ли́къ, сті́хъ:}
И҆ да́стъ крѣ́пость царю̀:
\hKv{и҆ вторы́й тропа́рь канѡ́на мине́и. Та́же,}
Сла́ва, и҆  ны́нѣ:
\hKv{съ тропарѝ канѡ́на. И҆ пое́тъ пе́рвый ли́къ  і҆рмо́съ мине́и тре́тїѧ пѣ́сни, послѣдѝ канѡ́на. Си́мъ  ѡ҆́бразомъ пое́мъ и҆ ѕ҃_ю пѣ́снь. 
Четве́ртꙋю же и҆ пѧ́тꙋю,  и҆ седмꙋ́ю пѣ̑сни си́це пое́мъ: Глаго́ли пе́рвѣе і҆рмо́съ  канѡ́на мине́и. 
Та́же конє́чныѧ два̀ стїха̑ пѣ́сни пое́мъ  къ тропарє́мъ канѡ́на. и҆}
Сла́ва, и҆ ны́нѣ: 
\hKv{І҆рмосо́мъ же  не покрыва́емъ пѣ̑сни, занѐ і҆рмо́съ глаго́летсѧ пре́жде  пѣ́сни. Пою́тсѧ же   і҆рмосы̀ мине́и по канѡ́нѣ, то́кмѡ по г҃_й и҆ ѕ҃_й  пѣ́снехъ. Зане́же ᲂу҆ тѣ́хъ пѣ́сней пе́рвѣе і҆рмо́съ не  пое́тсѧ. Въ про́чыѧ же дни̑ седми́цы, кромѣ̀  понедѣ́льника, та́кѡ пое́тсѧ и҆ пе́рваѧ пѣ́снь, ꙗ҆́коже  ᲂу҆каза́сѧ ѡ҆ д҃-й и҆ є҃-й пѣ́снехъ. А҆ въ ни́хже  пѣ́снехъ бꙋ́детъ трипѣ́снецъ, пое́тсѧ послѣдѝ канѡ́на  катава́сїа, і҆рмо́съ трипѣ́снца, ꙗ҆́коже вы́ше  и҆з̾ѧви́сѧ.} 

\hKv{Вѣ́домо же бꙋ́ди, ꙗ҆́кѡ ѻ҆смогла́сникъ не пое́тсѧ во всю̀  ст҃ꙋ́ю м҃_цꙋ, кромѣ̀ недѣ́ль. Пое́тсѧ же то́чїю мине́а, и҆  трипѣ́снецъ. А҆ въ ни́хже пѣ́снехъ нѣ́сть трипѣ́снца,  пое́тсѧ то́кмѡ мине́а.}  

