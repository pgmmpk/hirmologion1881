\centergraphics[width=7cm]{p-082-tone4.pdf}

\cuKinovarCenter{{\ponomarexpandedfont Нача́ло і҆рмосѡ́въ є҃_гѡ гла́са.}}

\cuSec{Пѣ́снь а҃.}{Гла́съ ѕ҃.}
\label{tone.6}

\vskip-1.2em
\cuLettrine
Ꙗ҆́кѡ по сꙋ́хꙋ пѣшеше́ствовавъ і҆и҃ль, по бе́зднѣ  стопа́ми, гони́телѧ фараѡ́на ви́дѧ потоплѧ́ема, бг҃ꙋ  побѣ́днꙋю пѣ́снь пои́мъ, вопїѧ́ше. 
%
\cuSubSec{Въ вели́кїй четверто́къ:}

\hKv Сѣ́ченное сѣче́тсѧ мо́ре  чермно́е, волнопита́емаѧ же и҆зсꙋша́етъ глꙋбина̀, та́ѧжде  кꙋ́пнѡ без̾ѻрꙋ̑жнымъ бы́вши проходи́ма, и҆ всеѻрꙋ́жнымъ  гро́бъ. пѣ́снь же бг҃окра́снаѧ воспѣва́шесѧ: сла́внѡ  просла́висѧ хрⷭ҇то́съ бг҃ъ на́шъ. 
%
\cuSubSec{Въ вели́кꙋю сꙋббѡ́тꙋ:}

\hKv Волно́ю морско́ю скры́вшаго дре́вле  гони́телѧ мꙋчи́телѧ, под̾ земле́ю скры́ша, сп҃се́нныхъ  ѻ҆́троцы: но мы̀   ꙗ҆́кѡ  ѻ҆трокови̑цы, гдⷭ҇еви пои́мъ, сла́внѡ бо просла́висѧ. 
%
\cuSubSec{Ѻ҆смогла́сника:}

\hKv Чꙋ́вственный фараѡ́нъ потопле́нъ бы́сть со всево́инствомъ,  і҆и҃ль же проше́дъ посредѣ̀ мо́рѧ, вопїѧ́ше: гдⷭ҇еви бг҃ꙋ  на́шемꙋ пои́мъ, ꙗ҆́кѡ просла́висѧ. 

\hKv Помо́щникъ и҆ покрови́тель бы́сть мнѣ̀ во сп҃се́нїе се́й  мо́й бг҃ъ, и҆ просла́влю є҆го̀, бг҃ъ ѻ҆ц҃а̀ моегѡ̀, и҆  вознесꙋ̀ є҆го̀, сла́внѡ бо просла́висѧ. 

\hKv Пои́мъ гдⷭ҇еви, ѿ рабо́ты і҆и҃лѧ свободи́вшемꙋ  є҆гѵ́петскїѧ мѡѷсе́емъ, пѣ́снь непреста́ннѡ хрⷭ҇тꙋ̀ цр҃ю̀  и҆ бг҃ꙋ на́шемꙋ. 

\hKv Проше́дше мо́ре сы́нове і҆и҃лєвы, и҆ ви́дѧще фараѡ́на  погрꙋже́на въ не́мъ, бл҃года́рнѡ поѧ́хꙋ: гдⷭ҇еви пои́мъ,  сла́внѡ бо просла́висѧ. 

\hKv Воздвиза́емое мо́ре немо́крыми нога́ми проходѧ̀ дре́вле  і҆и҃ль, є҆гда̀ ви́дѧше погрꙋжа́ема го́рдаго фараѡ́на,  ра́достнѡ взыва́ше: пои́мъ гдⷭ҇еви сотво́ршемꙋ сла̑внаѧ  чꙋдеса̀.  

\hKv Колесни̑цы фараѡ́нѡвы, и҆ си́лꙋ є҆гѡ̀, высо́кою мы́шцею  хрⷭ҇то́съ потопѝ, і҆и҃лѧ же сп҃сѐ воспѣва́юща: ди́вномꙋ  бг҃ꙋ на́шемꙋ пои́мъ. 
%
\cuSec{Пѣ́снь в҃.}{Гла́съ ѕ҃.}

\hKv Вонмѝ нб҃о, и҆ возглаго́лю, и҆ воспою̀ хрⷭ҇та̀, ѿ дв҃ы  пло́тїю прише́дшаго. 

\hKv Ви́дите ви́дите, ꙗ҆́кѡ а҆́зъ є҆́смь бг҃ъ, ма́ннꙋ  ѡ҆дожди́вый, и҆ во́дꙋ и҆з̾ ка́мене и҆сточи́вый дре́вле въ  пꙋсты́ни лю́демъ мои̑мъ, десни́цею є҆ди́ною и҆ крѣ́постїю  мое́ю. 

\hKv Вонмѝ нб҃о, и҆ возглаго́лю, и҆ восхвалю̀ хрⷭ҇та̀,  є҆ди́наго чл҃вѣколю́бца. 

\hKv Ви́дите ви́дите, ꙗ҆́кѡ а҆́зъ є҆́смь бг҃ъ ва́шъ, крⷭ҇то́мъ  а҆́дъ разори́вый ꙗ҆́кѡ си́ленъ, и҆ безстра̑стїѧ стра́стїю  пло́ти своеѧ̀ и҆сточи́вый земны̑мъ.  
%
\cuSec{Пѣ́снь г҃.}{Гла́съ ѕ҃.}

\hKv Нѣ́сть ст҃ъ, ꙗ҆́коже ты̀ гдⷭ҇и бж҃е мо́й, вознесы́й ро́гъ  вѣ́рныхъ твои́хъ бл҃же, и҆ ᲂу҆тверди́вый на́съ на ка́мени  и҆сповѣ́данїѧ твоегѡ̀. 
%
\cuSubSec{Въ вели́кїй четверто́къ:}

\hKv Гдⷭ҇ь сы́й всѣ́хъ, и҆ зижди́тель  бг҃ъ, созда́нное безстра́стный ѡ҆бнища́въ себѣ̀ соединѝ,  и҆ па́сха за ꙗ҆́же хотѧ́ше ᲂу҆мре́ти, са́мъ сы́й себѐ  предпожрѐ: ꙗ҆ди́те, вопїѧ̀, тѣ́ло моѐ, и҆ вѣ́рою  ᲂу҆тверди́тесѧ. 
%
\cuSubSec{Въ вели́кꙋю сꙋббѡ́тꙋ:}

\hKv Тебѐ на вода́хъ повѣ́сившаго всю̀  зе́млю неѡдержи́мѡ, тва́рь ви́дѣвши на ло́бнѣмъ ви́сима,  ᲂу҆́жасомъ мно́гимъ содрага́шесѧ, нѣ́сть ст҃ъ, ра́звѣ  тебѐ гдⷭ҇и взыва́ющи. 
%
\cuSubSec{Ѻ҆смогла́сника:}

\hKv На тве́рдѣмъ вѣ́ры твоеѧ̀ ка́мени, помышле́нїе  ᲂу҆тверди́въ дꙋшѝ моеѧ̀, ᲂу҆твердѝ гдⷭ҇и: тѧ́ бо и҆́мамъ  бл҃же, прибѣ́жище и҆ ᲂу҆твержде́нїе.  

\hKv Оу҆твердѝ гдⷭ҇и на ка́мени за́повѣдей твои́хъ  подви́гшеесѧ се́рдце моѐ, ꙗ҆́кѡ є҆ди́нъ ст҃ъ є҆сѝ и҆  гдⷭ҇ь. 

\hKv На недвижи́момъ хрⷭ҇тѐ, ка́мени за́повѣдей твои́хъ,  ᲂу҆твердѝ моѐ помышле́нїе. 

\hKv Оу҆твердѝ гдⷭ҇и цр҃ковь твою̀, ра́зꙋмомъ возвы́сивый  нб҃са̀, во є҆́же пѣ́ти пречⷭ҇тое твоѐ смотре́нїе,  є҆ди́не чл҃вѣколю́бче.  

\hKv Гдⷭ҇и сп҃се мо́й и҆ зижди́телю бж҃е всѣ́хъ, зе́млю на  ничесо́мъ повелѣ́нїемъ, въ нача́лѣ тѧ́жестнꙋ  ᲂу҆тверди́вый, ᲂу҆твердѝ и҆ менѐ въ вѣ́рѣ, є҆́же  вопи́ти тебѣ̀ въ весе́лїи: ты̀ моѧ̀ крѣ́пость и҆ си́ла,  ѻ҆рꙋ́жїе и҆ прибѣ́жище. 

\hKv Оу҆твердѝ менѐ хрⷭ҇тѐ и҆ и҆звѣсти́ мѧ къ тебѣ̀, и҆  просвѣще́нїемъ ра́зꙋма твоегѡ̀ наꙋчи́ мѧ. 

\hKv Разшири́шасѧ на врагѝ моѧ̑ ᲂу҆ста̀ моѧ̑, ꙗ҆́кѡ  ᲂу҆тверди́сѧ ѡ҆ гдⷭ҇ѣ се́рдце моѐ. 

\hKv Тобо́ю хрⷭ҇тѐ нб҃са̀ ᲂу҆твержда́єма всѧ̑, сло́во бж҃їе и҆  си́ла, и҆сповѣ́дꙋютъ сла́вꙋ неизрече́ннꙋ,  и҆ вседѣ́тельныѧ рꙋкѝ твоеѧ̀ сотворе́нїѧ:  нѣ́сть бо пресвѧ́тъ, ра́звѣ тебѐ гдⷭ҇и. 

\hKv Въ пе́рсѣхъ се́рдца и҆ дꙋшѝ моеѧ̀ зовꙋ̀ тебѣ̀, ꙗ҆́кѡ  прⷪ҇ро́чица а҆́нна: ꙗ҆́коже ѻ҆́ной да́лъ є҆сѝ мл҃твою  самꙋ́ила, и҆ мнѣ̀ да́рꙋй гдⷭ҇и, пло́дъ покаѧ́нїѧ. 

\hKv Возвесели́сѧ непло́ды неражда́ющаѧ, ꙗ҆́кѡ прїи́де  и҆зба́витель мо́й и҆ бг҃ъ, ѿ де́мѡнскїѧ пре́лести  и҆збавлѧ́ѧй. 

\hKv Неизрече́нныѧ твоеѧ̀ си́лы, и҆ содержи́тельныѧ всѣ́хъ  премꙋ́дрости твоеѧ̀, ᲂу҆страша́юсѧ помышлѧ́ѧй, и҆ вопїю́ ти  бл҃же: на врагѝ да вознесе́тсѧ хрⷭ҇тѐ ро́гъ мо́й. 
%
\cuSec{Пѣ́снь д҃.}{Гла́съ ѕ҃.}

\hKv Хрⷭ҇то́съ моѧ̀ си́ла, бг҃ъ и҆ гдⷭ҇ь, чⷭ҇тна́ѧ цр҃ковь  бг҃олѣ́пнѡ пое́тъ взыва́ющи, ѿ смы́сла чи́ста ѡ҆ гдⷭ҇ѣ  пра́зднꙋющи. 
%
\cuSubSec{Въ вели́кїй четверто́къ:}

\hKv Прови́дѣвъ прⷪ҇ро́къ та́йнꙋ  твою̀ неизрече́ннꙋю хрⷭ҇тѐ, провозгласѝ:  положи́лъ є҆сѝ тве́рдꙋю любо́вь крѣ́пости,  ѻ҆́ч҃е ще́дрый: є҆диноро́днаго бо сн҃а бл҃гі́й ѡ҆чище́нїе  въ мі́ръ посла́лъ є҆сѝ. 
%
\cuSubSec{Въ вели́кꙋю сꙋббѡ́тꙋ:}

\hKv На крⷭ҇тѣ̀ твоѐ бж҃е́ственное  и҆стоща́нїе прови́дѧ а҆ввакꙋ́мъ ᲂу҆жа́ссѧ вопїѧ́ше: ты̀  си́льныхъ пресѣ́клъ є҆сѝ держа́вꙋ бл҃же, прїѡбща́ѧсѧ  сꙋ́щымъ во а҆́дѣ, ꙗ҆́кѡ всеси́ленъ. 
%
\cuSubSec{Ѻ҆смогла́сника:}

\hKv Покры́ла є҆́сть нб҃са̀ добродѣ́тель твоѧ̀ хрⷭ҇тѐ, и҆  твоегѡ̀ хвале́нїѧ гдⷭ҇и, всѧ̑ и҆спо́лнишасѧ. 

\hKv Оу҆диви́сѧ тво́й ра́зꙋмъ ѿ менѐ, ᲂу҆слы́шавшꙋ  пресла̑внаѧ велѣ̑нїѧ смотре́нїѧ твоегѡ̀, ᲂу҆крѣпи́сѧ же  возлюбле́нїемъ твоегѡ̀ схожде́нїѧ: ты́ бо моеѧ̀ нищеты̀ не  ѿве́рглсѧ є҆сѝ. 

\hKv Оу҆слы́шавъ повелѣ́нїѧ, презрѣ́хъ гдⷭ҇и, и҆ пꙋтѝ твоегѡ̀  не сохрани́въ ᲂу҆мертви́хсѧ: ты́ мѧ возста́ви сп҃се  наде́ждо моѧ̀, менѐ ра́ди ѡ҆бнища́вый пло́ть человѣ́ческꙋю  прїи́мъ.  

\hKv Оу҆слы́шахъ слꙋ́хъ тво́й, и҆ ᲂу҆боѧ́хсѧ, разꙋмѣ́хъ дѣла̀  твоѧ̑, и҆ ᲂу҆жасо́хсѧ, гдⷭ҇и. 

\hKv Оу҆слы́шахъ слꙋ́хъ тво́й, и҆ ᲂу҆боѧ́хсѧ, разꙋмѣ́хъ дѣла̀  твоѧ̑ и҆ ᲂу҆жасо́хсѧ: сла́ва си́лѣ твое́й гдⷭ҇и. 

\hKv Оу҆слы́шахъ прⷪ҇ро́къ прише́ствїе твоѐ гдⷭ҇и, и҆  ᲂу҆боѧ́сѧ, ꙗ҆́кѡ хо́щеши ѿ дв҃ы роди́тисѧ, и҆  человѣ́кѡмъ ꙗ҆ви́тисѧ, и҆ глаго́лаше: ᲂу҆слы́шахъ слꙋ́хъ  тво́й, и҆ ᲂу҆боѧ́хсѧ, сла́ва си́лѣ твое́й гдⷭ҇и. 

\hKv Ѡ҆бнови́лъ є҆сѝ добро́тꙋ твоегѡ̀ ѡ҆́браза, къ семꙋ́ бо ѿ  ѻ҆ч҃ескихъ соше́лъ є҆сѝ нѣ́дръ, и҆ вопїе́мъ тѝ: сла́ва  си́лѣ твое́й, чл҃вѣколю́бче. 

\hKv Твоегѡ̀ неизрече́ннагѡ воплоще́нїѧ, и҆зба́вителю всѣ́хъ,  прⷪ҇ро́къ а҆ввакꙋ́мъ, ѡ҆чи́стивсѧ дх҃омъ, та́инствꙋ  ᲂу҆жа́ссѧ, вопїѧ́ше: внегда̀ прибли́жатсѧ лѣ̑та, позна́нъ  бꙋ́деши, безма́теренъ ѿ ѻ҆ц҃а̀, въ послѣ̑днѧѧ времена̀  воплоща́ѧйсѧ.  

\hKv Взы́дꙋ на высотꙋ̀ хвале́нїѧ твоегѡ̀, и҆ лю́демъ и҆збра̑ннымъ  бл҃говѣщꙋ̀ ᲂу҆покое́нїе ѿ ѡ҆сꙋжде́нїѧ снѣ́ди, прⷪ҇ро́къ  глаго́лаше: сло́во бо ѡ҆дебелѣ́вшее, ѿ дв҃ы на́мъ  ꙗ҆ви́сѧ. 
%
\cuSec{Пѣ́снь є҃.}{Гла́съ ѕ҃.}

\hKv Бж҃їимъ свѣ́томъ твои́мъ бл҃же, ᲂу҆́треннюющихъ тѝ дꙋ́шы  любо́вїю ѡ҆зарѝ, молю́сѧ, тѧ̀ вѣ́дѣти сло́ве бж҃їй,  и҆́стиннаго бг҃а, ѿ мра́ка грѣхо́внагѡ взыва́юща. 
%
\cuSubSec{Въ вели́кїй четверто́къ:}

\hKv Сою́зомъ любвѐ свѧзꙋ́еми  а҆пⷭ҇ли, влады́чествꙋющемꙋ всѣ́ми себѐ хрⷭ҇тꙋ̀ возло́жше:  кра̑сны но́ги ѡ҆чища́хъ бл҃говѣствꙋ́юще всѣ̑мъ ми́ръ. 
%
\cuSubSec{Въ вели́кїй пѧто́къ:}

\hKv Къ тебѣ̀ ᲂу҆́треннюю: милосе́рдїѧ  ра́ди себѐ и҆стощи́вшемꙋ непрело́жнѡ, и҆ до страсте́й  безстра́стнѡ прекло́ньшемꙋсѧ сло́ве бж҃їй: ми́ръ пода́ждь  мѝ па́дшемꙋ чл҃вѣколю́бче.  
%
\cuSubSec{Въ вели́кꙋю сꙋббѡ́тꙋ:}

\hKv Бг҃оѧвле́нїѧ твоегѡ̀ хрⷭ҇тѐ, къ  на́мъ ми́лостивнѡ бы́вшагѡ, и҆са́їа свѣ́тъ ви́дѣвъ  невече́рнїй, и҆з̾ но́щи ᲂу҆́треневавъ взыва́ше:  воскре́снꙋтъ ме́ртвїи, и҆ воста́нꙋтъ сꙋ́щїи во гробѣ́хъ, и҆  всѝ земноро́днїи возра́дꙋютсѧ. 
%
\cuSubSec{Ѻ҆смогла́сника:}

\hKv Свѣ́тъ возсїѧ́вый мі́рꙋ хрⷭ҇тѐ, просвѣтѝ се́рдце моѐ  ѿ но́щи тебѣ̀ зовꙋ́ща, и҆ сп҃си́ мѧ. 

\hKv На тѧ̀ ᲂу҆пова́хъ гдⷭ҇и, и҆ къ тебѣ̀ ᲂу҆́тренюю  пребж҃е́ственнѣй добро́тѣ, дꙋ́шꙋ мою̀ возвеселѝ во  свѣ́тѣ бг҃овѣ́дѣнїѧ твоегѡ̀, и҆ сп҃си́ мѧ. 

\hKv Прозорли́вый и҆са́їа, разꙋмѣ́ѧй безсѣ́менное рожде́нїе  твоѐ дв҃о, вопїѧ́ше: сѐ во чре́вѣ прїи́метъ дв҃а,  ро́ждшаѧ сло́ва є҆диноро́днагѡ, сн҃а же ѻ҆́ч҃а, со дх҃омъ  бж҃е́ственнымъ славосло́вима. 

\hKv Ѿ но́щи пѣ́снь возсыла́емъ тебѣ̀, возниспослѝ на́мъ  ми́лость твою̀ гдⷭ҇и: ты́ бо є҆сѝ бг҃ъ на́шъ, ра́звѣ  тебѣ̀ и҆но́гѡ не вѣ́мы.  

\hKv Ѿ но́щи ᲂу҆́тренююща, чл҃вѣколю́бче просвѣтѝ, молю́сѧ,  и҆ наста́ви и҆ менѐ на повелѣ̑нїѧ твоѧ̑, и҆ наꙋчи́ мѧ  сп҃се, твори́ти во́лю твою̀. 

\hKv Свѣ́тъ и҆ ми́ръ тво́й хрⷭ҇тѐ, рабѡ́мъ твои̑мъ пода́ждь  бл҃же: ты́ бо є҆сѝ всѣ́хъ ми́ръ, и҆ сою́зъ любвѐ  чл҃вѣколю́бче, тѧ̀ позна́вшымъ ᲂу҆́тро свѣ́тлое, и҆  влⷣкꙋ вои́стиннꙋ и҆ гдⷭ҇а. 

\hKv Мра́къ тѧ̀, дꙋшѐ моѧ̀, поѧ́тъ беззако́нныхъ дѣѧ́нїй,  ꙗ҆̀же содѣ́лала є҆сѝ въ житїѝ се́мъ, но не преста́ни  зовꙋ́щи: наста́вниче свѣ́та, и҆ и҆зба́вителю всѣ́хъ,  ѡ҆мраче́ннꙋю дꙋ́шꙋ мою̀ свѣтово́дствꙋй къ свѣ́тꙋ  невече́рнемꙋ. 

\hKv Положѝ на стѡпы̀ моѧ̑ мглꙋ̀ всеѕлѣ́йшїй вра́гъ, гдⷭ҇и, но  ты̀ ѡ҆зарѝ свѣтода́вче сло́ве, и҆ сегѡ́ мѧ и҆зба́ви  ѡ҆сꙋжде́нїѧ. 

\hKv Ми́ръ мно́гъ лю́бѧщымъ тѧ̀ хрⷭ҇тѐ, и҆ но́щь просвѣще́нїѧ  въ пи́щи слове́съ твои́хъ: сегѡ̀ ра́ди ѿ но́щи  ᲂу҆́тренююще, про́симъ ми́лости ѿ тебѐ чл҃вѣколю́бче.   
%
\cuSec{Пѣ́снь ѕ҃.}{Гла́съ ѕ҃.}

\hKv Жите́йское мо́ре, воздвиза́емое зрѧ̀ напа́стей бꙋ́рею, къ  ти́хомꙋ приста́нищꙋ твоемꙋ̀ прите́къ, вопїю́ ти: возведѝ  ѿ тлѝ живо́тъ мо́й, многомлⷭ҇тиве. 
%
\cuSubSec{Въ вели́кїй четверто́къ:}

\hKv Бе́здна послѣ́днѧѧ грѣхѡ́въ  ѡ҆бы́де мѧ̀, и҆ волне́нїѧ не ктомꙋ̀ терпѧ̀, ꙗ҆́кѡ і҆ѡ́на  влⷣцѣ вопїю́ ти: ѿ тлѝ мѧ̀ возведѝ. 
%
\cuSubSec{Въ вели́кꙋю сꙋббѡ́тꙋ:}

\hKv Ꙗ҆́тъ бы́сть, но не ᲂу҆держа́нъ въ  пе́рсехъ ки́товыхъ і҆ѡ́на: тво́й бо ѡ҆́бразъ носѧ̀,  страда́вшагѡ, и҆ погребе́нїю да́вшагѡсѧ, ꙗ҆́кѡ ѿ черто́га  ѿ ѕвѣ́рѧ и҆зы́де. приглаша́ше же кꙋстѡді́и, хранѧ́щїи  сꙋ́єтнаѧ и҆ лѡ́жнаѧ, млⷭ҇ть сїю̀ ѡ҆ста́вили є҆стѐ. 
%
\cuSubSec{Ѻ҆смогла́сника:}

\hKv Ки́томъ пожре́нъ грѣхо́внымъ, вопїю̀ тебѣ̀ хрⷭ҇тѐ: ꙗ҆́кѡ  прⷪ҇ро́ка и҆з̾ и҆стлѣ́нїѧ мѧ̀ свободѝ.  

\hKv Ѡ҆бы́де мѧ̀ бе́здна послѣ́днѧѧ грѣхо́внаѧ пожре́ мѧ  мно́жество ѕѡ́лъ мои́хъ, и҆ стенѧ̀ вопїю́ ти бж҃е мо́й:  ꙗ҆́кѡ прⷪ҇ро́ка і҆ѡ́нꙋ и҆зба́ви многомлⷭ҇тиве. 

\hKv Ки́томъ содержи́мь прегрѣше́нїй, вопїю́ ти хрⷭ҇тѐ мо́й со  прⷪ҇ро́комъ: возведѝ ѿ тлѝ живо́тъ мо́й, и҆ сп҃си́ мѧ. 

\hKv Возопи́хъ всѣ́мъ се́рдцемъ мои́мъ къ ще́дромꙋ бг҃ꙋ, и҆  ᲂу҆слы́ша мѧ̀ ѿ а҆́да преиспо́днѧгѡ, и҆ возведѐ ѿ тлѝ  живо́тъ мо́й. 

\hKv Ѕвѣ́рю бе́здны предае́тсѧ прⷪ҇ро́къ і҆ѡ́на, глꙋбинѣ̀  морско́й, проѡбразꙋ́ѧ твоѐ сло́ве, тридне́вное воста́нїе.  тѣ́мъ и҆ взыва́ше: и҆з̾ глꙋбины̀ преиспо́днїѧ возведѝ  живо́тъ мо́й, бл҃гоꙋтро́бне. 

\hKv Возопи́хъ всѣ́мъ се́рдцемъ гдⷭ҇и, ѕлы́ми дѣ́лы ѡ҆держи́мь,  возведѝ, молю́сѧ, и҆зринове́на ѿ тлѝ: тѧ́ бо помѧнꙋ́хъ  сп҃се, и҆ возвесели́сѧ се́рдце моѐ.  
%
\cuSec{Пѣ́снь з҃.}{Гла́съ ѕ҃.}

\hKv Росода́тельнꙋ ᲂу҆́бѡ пе́щь содѣ́ла а҆́гг҃лъ преподѡ́бнымъ  ѻ҆трокѡ́мъ, халдє́и же ѡ҆палѧ́ющее велѣ́нїе бж҃їе  мꙋчи́телѧ ᲂу҆вѣща̀ вопи́ти: бл҃гослове́нъ є҆сѝ бж҃е  ѻ҆тє́цъ на́шихъ. 
%
\cuSubSec{Въ вели́кїй четверто́къ:}

\hKv Ѻ҆́троцы въ вавѷлѡ́нѣ пе́щнагѡ  пла́мене не ᲂу҆боѧ́шасѧ, но посредѣ̀ пла́мене вве́ржени,  ѡ҆роша́еми поѧ́хꙋ: бл҃гослове́нъ є҆сѝ гдⷭ҇и бж҃е ѻ҆тє́цъ  на́шихъ. 
%
\cuSubSec{Въ вели́кꙋю сꙋббѡ́тꙋ:}

\hKv Неизрече́нное чꙋ́до, въ пещѝ  и҆зба́вивый преподѡ́бныѧ ѻ҆́троки и҆з̾ пла́мене, во  гро́бѣ ме́ртвъ бездыха́ненъ полага́етсѧ, во сп҃се́нїе на́съ  пою́щихъ: и҆зба́вителю бж҃е бл҃гослове́нъ є҆сѝ. 
%
\cuSubSec{Ѻ҆смогла́сника:}

\hKv Преподо́бныхъ твои́хъ ѻ҆трокѡ́въ пѣ́снь ᲂу҆слы́шавый, и҆  пе́щь горѧ́щꙋю ѡ҆роси́вый, бл҃гослове́нъ є҆сѝ гдⷭ҇и бж҃е  ѻ҆тє́цъ на́шихъ.  

\hKv Прпⷣбныхъ твои́хъ ѻ҆трокѡ́въ мл҃твꙋ ᲂу҆слы́шавый, и҆  пе́щь горѧ́щꙋю ѡ҆роси́вый, бл҃гослове́нъ є҆сѝ гдⷭ҇и бж҃е  ѻ҆тє́цъ на́шихъ. 

\hKv Ты̀ ѻ҆́троки въ пещѝ жизнода́вче, а҆́гг҃ломъ ѡ҆роси́лъ  є҆сѝ, глаго́лющыѧ: бл҃гослове́нъ є҆сѝ гдⷭ҇и бж҃е  ѻ҆тє́цъ на́шихъ. 

\hKv Ꙗ҆ви́сѧ на ны̀ бл҃года́ть твоѧ̀ сп҃се, и҆ свѣ́тъ крⷭ҇та̀  твоегѡ̀ возсїѧ̀ мі́рови: бл҃гослове́нъ є҆сѝ гдⷭ҇и бж҃е  ѻ҆те́цъ на́шихъ. 

\hKv Согрѣши́хомъ, беззако́нновахомъ, непра́вдовахомъ пред̾  тобо́ю, нижѐ соблюдо́хомъ, нижѐ сотвори́хомъ, ꙗ҆́коже  заповѣ́далъ є҆сѝ на́мъ: но не преда́ждь на́съ до конца̀,  ѻ҆тцє́въ бж҃е. 

\hKv Трѝ ю҆́нѡши не ѡ҆палѝ дре́вле возвыша́емый пла́мень,  неща́днѡ мѧ̀ зно́й къ тлѣ́нїю положѝ. но ꙗ҆́кѡ ѻ҆́ныѧ  и҆зба́вилъ є҆сѝ сїѧ́нїемъ твои́мъ, пою́щаго съ ни́ми  сп҃си́ мѧ: бл҃гослове́нъ є҆сѝ бж҃е ѻ҆тє́цъ на́шихъ. 

\hKv Въ вавѷлѡ́нѣ плѣне́нїи тричи́сленнїи числе́нїемъ, царю̀  беззако́нномꙋ   проти́вишасѧ,  поклоне́нїю ѡ҆́браза дꙋшетли́тельнагѡ позлаще́ннагѡ. тѣ́мже  не покори́вшесѧ, въ пе́щь ѻ҆́гненнꙋю вве́ржени бы́ша, и҆  пою́ще поѧ́хꙋ: ѻ҆тє́цъ бж҃е бл҃гослове́нъ є҆сѝ. 

\hKv Въ пꙋсты́ни кꙋпинꙋ̀ неѡпа́льнѡ горѧ́щꙋю ѡ҆бра́знѡ  є҆вре́йскꙋю пе́щь, прописꙋ́ющꙋ показа̀, и҆ дв҃ы ѡ҆́бразъ  ꙗ҆влѧ́ѧй, чꙋде́съ бж҃е, и҆ препросла́вленный. 
%
\cuSec{Пѣ́снь и҃.}{Гла́съ ѕ҃.}

\hKv И҆з̾ пла́мене преподѡ́бнымъ ро́сꙋ и҆сточи́лъ є҆сѝ, и҆  пра́веднагѡ же́ртвꙋ водо́ю попали́лъ є҆сѝ: всѧ̑ бо  твори́ши хрⷭ҇тѐ, то́кмѡ є҆́же хотѣ́ти. тѧ̀ превозно́симъ  во всѧ̑ вѣ́ки. 
%
\cuSubSec{Въ вели́кїй четверто́къ:}

\hKv За зако́ны ѻ҆те́чєскїѧ  бл҃же́ннїи въ вавѷлѡ́нѣ ю҆́ношы предбѣ́дствꙋюще,  царю́ющагѡ ѡ҆плева́ша повелѣ́нїе безꙋ́мное, и҆ совокꙋ́плени  и҆́мже не свари́шасѧ ѻ҆гне́мъ, держа́вствꙋюшемꙋ досто́йнꙋю   воспѣва́хꙋ пѣ́снь: гдⷭ҇а по́йте  дѣла̀, и҆ превозноси́те во всѧ̑ вѣ́ки. 
%
\cuSubSec{Въ вели́кїй пѧто́къ:}

\hKv Сто́лпъ ѕло́бы бг҃опроти́вныѧ  бж҃е́ственнїи ѻ҆́троцы ѡ҆бличи́ша, на хрⷭ҇та̀ шата́ющеесѧ  беззако́нныхъ собо́рище, совѣ́тꙋетъ тщє́тнаѧ, ᲂу҆би́ти  поꙋча́етсѧ живо́тъ держа́щаго дла́нїю. є҆го́же всѧ̀ тва́рь  бл҃гослови́тъ, сла́вѧщи во вѣ́ки. 
%
\cuSubSec{Въ вели́кꙋю сꙋббѡ́тꙋ:}

\hKv Оу҆жасни́сѧ боѧ́йсѧ нб҃о, и҆ да  подви́жатсѧ ѡ҆снова̑нїѧ землѝ, се́ бо въ мертвецѣ́хъ  вмѣнѧ́етсѧ въ вы́шнихъ живы́й, и҆ во гро́бъ ма́лъ  страннопрїе́млетсѧ є҆го́же ѻ҆́троцы бл҃гослови́те,  сщ҃е́нницы воспо́йте, лю́дїе превозноси́те во всѧ̑ вѣ́ки. 
%
\cuSubSec{Ѻ҆смогла́сника:}

\hKv Прпⷣбнїи твоѝ ѻ҆́троцы въ пещѝ херꙋві́мы подража́хꙋ,  трист҃ꙋ́ю пѣ́снь воспѣва́юще: бл҃гослови́те, по́йте и҆  превозноси́те въ всѧ̑ вѣ́ки.  

\hKv Ѻ҆трокѡ́въ пѣ́снь воспои́мъ хрⷭ҇тꙋ̀, пою́ще съ ни́ми. да  бл҃гослови́тъ всѧ̀ тва́рь гдⷭ҇а, и҆ превозно́ситъ во всѧ̑  вѣ́ки. 

\hKv Є҆го́же вѡ́инства нбⷭ҇наѧ сла́вѧтъ, и҆ трепе́щꙋтъ херꙋві́ми  и҆ серафі́ми, всѧ́ко дыха́нїе и҆ тва́рь, по́йте,  бл҃гослови́те и҆ превозноси́те во всѧ̑ вѣ́ки. 

\hKv Мꙋсїкі́йскагѡ согла́сїѧ небре́гше ѻ҆́троцы, пѣ́снь  бж҃е́ственнꙋю и҆з̾ пла́мене воспѣва́хꙋ, глаго́люще:  сщ҃е́нницы бл҃гослови́те, лю́дїе, кѡлѣ́на и҆ ꙗ҆зы́цы  превозноси́те гдⷭ҇а. 

\hKv Бг҃а ѻ҆ц҃а̀ содѣ́телѧ, є҆диносꙋ́щна сн҃а, и҆ бж҃їѧ дх҃а,  въ пла́мени ю҆́нѡши пѣ́ти повелѣва́ютсѧ: да бл҃гослови́тъ  гдⷭ҇а всѧ̀ тва́рь, и҆ превозно́ситъ во всѧ̑ вѣ́ки. 

\hKv Мꙋчи́телевꙋ златослїѧ́нномꙋ столпꙋ̀, ꙗ҆́кѡ  сопротивобо́жномꙋ и҆стꙋка́нꙋ, не поклони́шасѧ ѻ҆́троцы  сїѡ́нстїи, но бг҃охрани́ми, персі́йскꙋю пе́щь ꙗ҆́кѡ  травни́къ непщева́хꙋ, и҆ пла́мень ꙗ҆́кѡ ѡ҆роша́ющїй  ѻ҆́блакъ, и҆  ликꙋ́юще поѧ́хꙋ:  бл҃гослови́те творє́нїѧ всѧ̑ гдⷭ҇а. 

\hKv Оу҆грожа́ѧй ца́рь и҆ногда̀ ю҆́ношамъ, пе́щь ᲂу҆гото́ва, и҆  непови̑ннымъ повелѣ̀ вве́ржєнымъ бы́ти, и҆сповѣ́дающымъ  бг҃а препросла́вленнаго: є҆го́же бл҃гословѧ́тъ творє́нїѧ  всеѧ̀ тва́ри во вѣ́ки. 

\hKv Си́ла та́инственнаѧ въ вавѷлѡ́нѣ ꙗ҆влѧ́етсѧ, ѻ҆́троки въ  пла́мени бж҃е́ственнымъ дх҃омъ ѡ҆роша́ющи, и҆ мꙋчи́телевꙋ  горды́ню попалѧ́ющи. тѣ́мже вопїѧ́хꙋ бл҃гоче́стнѡ:  бл҃гослови́те дѣла̀ гдⷭ҇нѧ гдⷭ҇а. 

\hKv Цр҃ю̀ вѣкѡ́въ хрⷭ҇тꙋ̀, пѣ́снь рце́мъ ѻ҆трокѡ́въ вѣ́рнїи:  бл҃гослови́те всѧ̑ дѣла̀ гдⷭ҇нѧ гдⷭ҇а. 

\hKv Прохлажде́й горѧ́щꙋю пе́щь бж҃е, и҆ ѻ҆́троки же невреди́мы  сохрани́въ тебѐ воспѣва́ющыѧ: сщ҃е́нницы по́йте, и҆  превозноси́те є҆го̀ во вѣ́ки.  
%
\cuSec{Пѣ́снь ѳ҃.}{Гла́съ ѕ҃.}

\hKv Бг҃а человѣ́кѡмъ не возмо́жно ви́дѣти, на него́же не  смѣ́ютъ чи́ни а҆́гг҃льстїи взира́ти: тобо́ю же всечⷭ҇таѧ  ꙗ҆ви́сѧ человѣ́кѡмъ сло́во воплоще́нно, є҆го́же  велича́юще съ нбⷭ҇ными вѡ́и, тѧ̀ ᲂу҆бл҃жа́емъ. 
%
\cuSubSec{Въ вели́кїй четверто́къ:}

\hKv Стра́нствїѧ влⷣчнѧ, и҆  безсме́ртныѧ трапе́зы на го́рнемъ мѣ́стѣ, высо́кими  ᲂу҆мы̀, вѣ́рнїи прїиди́те наслади́мсѧ, возше́дша сло́ва,  ѿ сло́ва наꙋчи́вшесѧ, є҆го́же велича́емъ. 
%
\cuSubSec{Въ вели́кїй пѧто́къ:}

\hKv Чⷭ҇тнѣ́йшꙋю херꙋві̑мъ, и҆  сла́внѣйшꙋю без̾ сравне́нїѧ серафі̑мъ, без̾ и҆стлѣ́нїѧ  бг҃а сло́ва ро́ждшꙋю, сꙋ́щꙋю бцⷣꙋ тѧ̀ велича́емъ. 
%
\cuSubSec{Въ вели́кꙋю сꙋббѡ́тꙋ:}

\hKv Не рыда́й менѐ мт҃и зрѧ́щи во  гро́бѣ, є҆го́же во чре́вѣ без̾ сѣ́мене зачала̀ є҆сѝ  сн҃а: воста́нꙋ бо и҆ просла́влюсѧ, и҆ вознесꙋ̀ со сла́вою  непреста́ннѡ ꙗ҆́кѡ бг҃ъ, вѣ́рою и҆ любо́вїю тѧ̀  велича́ющыѧ.  
%
\cuSubSec{Ѻ҆смогла́сника:}

\hKv Є҆́же ра́дꙋйсѧ ѿ а҆́гг҃ла прїи́мшаѧ, и҆ ро́ждшаѧ  созда́телѧ своего̀, дв҃о, сп҃си́ тѧ велича́ющыѧ. 

\hKv Вѣ́чное ѡ҆брѣ́тше и҆збавле́нїе, ѿ пра́ѻтца а҆да́ма  свы́ше рожде́нное, ро́дꙋ стра́шнагѡ и҆зрече́нїѧ, рождество̀  со безпло́тными непостижи́мое славосло́вимъ, бцⷣꙋ тѧ̀  пѣ́сньми велича́юще. 

\hKv Ра́дꙋйсѧ чⷭ҇тна́ѧ всест҃а́ѧ мт҃и дв҃о, є҆ди́на всепѣ́таѧ,  тѧ̀ сла́витъ всѧ̀ тва́рь, мт҃рь свѣ́та. 

\hKv Безсѣ́меннагѡ зача́тїѧ ржⷭ҇тво̀ несказа́нное, мт҃ре  безмꙋ́жныѧ нетлѣ́ненъ пло́дъ: бж҃їе бо рожде́нїе  ѡ҆бновлѧ́етъ є҆стєства̀. тѣ́мже тѧ̀ всѝ ро́ди, ꙗ҆́кѡ  бг҃оневѣ́стнꙋю мт҃рь, правосла́внѡ велича́емъ. 

\hKv Сло́вомъ сло́во є҆ди́на ро́ждшаѧ, є҆реті́чєскаѧ ᲂу҆ста̀  заградѝ ѕлорѣчи̑ваѧ. и҆́но но́во  нб҃о, неѡкра́домый ра́й пречⷭ҇таѧ, тѧ̀ велича́емъ. 

\hKv Велича́емъ тѧ̀ всѝ ро́ди бцⷣе дв҃о, и҆сто́чникъ живота̀  на́шегѡ. 

\hKv Сѐ врата̀, сѐ гора̀ ст҃а́ѧ, сѐ дѣ́вствꙋющаѧ мт҃и, ю҆́же  велича́емъ. 

\hKv Слы́ши дщѝ, и҆ ви́ждь, и҆ приклонѝ ᲂу҆́хо твоѐ, и҆  возжела́етъ ца́рь добро́тѣ твое́й: и҆́бо всѧ̀ сла́ва твоѧ̀  дв҃о ѿвнꙋ́трь, ꙗ҆́кѡ заче́нши созда́телѧ твоего̀. тѣ́мже  бога́тїи лю́дстїи лицꙋ̀ твоемꙋ̀ молѧ́щесѧ, тѧ̀ велича́емъ. 

\hKv Мт҃рними ѡ҆б̾ѧ́тїи чⷭ҇тна́ѧ носи́ла є҆сѝ сѣдѧ́щаго на  херꙋві́мѣхъ, и҆ сего̀ сосцепита́еши, за кра́йнюю всѧ́кꙋю  бл҃гость, во а҆да́мовъ ѡ҆́бразъ воѡбразꙋ́ющасѧ. тѣ́мже  соше́дшесѧ человѣ́ческое собра́нїе, и҆ тѧ̀ ро́ждшꙋю свѣ́тъ  мі́ра, велича́емъ. 

\hKv Безсѣ́меннѡ ро́ждшꙋю є҆ди́наго ѿ трⷪ҇цы на́ше всѐ без̾  и҆стлѣ́нїѧ взе́мша, є҆мꙋ́же   ѻ҆те́чєства ꙗ҆зы́кѡвъ покланѧ́ютсѧ, дв҃о тѧ̀ велича́емъ. 

\hKv Кꙋпинꙋ̀ ᲂу҆́мнꙋю дре́вле на горѣ̀ ꙗ҆́вльшꙋюсѧ мѡѷсе́ю,  неѡпа́льнѡ горѧ́щꙋ, пречⷭ҇тꙋю и҆ ст҃ꙋ́ю дв҃ꙋ, согла́снѡ  пѣ́сньми велича́емъ.  

