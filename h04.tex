\cleartorecto

\pagestyle{hirmologion-simplehead}

\centergraphics[width=6cm]{p-004-tochead.pdf}

\cuKinovarCenter{%
Ѡ҆главле́нїе веще́й ѡ҆брѣта́ющихсѧ въ кни́зѣ се́й, глаго́лемѣй і҆рмоло́гїй.
}%
\thispagestyle{hirmologion-simplehead}

І҆рмосы̀, гла́съ а҃, \hPageref{tone.1}.

Гла́съ в҃, \hPageref{tone.2}.

Гла́съ г҃, \hPageref{tone.3}. 

Гла́съ д҃, \hPageref{tone.4}.

Гла́съ є҃, \hPageref{tone.5}.

Гла́съ ѕ҃, \hPageref{tone.6}. 

Гла́съ з҃, \hPageref{tone.7}.

Гла́съ и҃, \hPageref{tone.8}.

І҆рмосы̀ въ предпра́зднство хрⷭ҇то́ва ржⷭ҇тва̀, \hPageref{xmas}. 

І҆рмосы̀ въ предпра́зднство ст҃ы́хъ бг҃оѧвле́нїй, \hPageref{epif}. 

Лїтꙋргі́а вседне́внаѧ, \hPageref{lit}. 

Гдⷭ҇и воззва́хъ, пое́мый на вече́рнѧхъ, и҆ лїтꙋргі́ахъ  преждеѡсщ҃е́нныхъ, \hPageref{litfty}. 

Бг҃оро́дичны воскре́сны, на ѻ҆́смь гласѡ́въ, \hPageref{bogorod}. 

Бг҃оро́дичны по всѧ̑ дни̑, пое́мїи на сла́вахъ ст҃ы́хъ, \hPageref{bogorod8}. 

Степє́нны, на ѻ҆́смь гласѡ́въ, \hPageref{step}.

Пѣ́сни трⷪ҇чны, на ѻ҆́смь гласѡ́въ, и҆ пѣ́сней стїхѝ пое́мїи  по всѧ̑ дни̑ къ канѡ́нѡмъ, \hPageref{trinity}.

Пѣ́сней стїхѝ, пое́мїи въ недѣ̑ли и҆ пра́здники. \hPageref{prazd}.

Оу҆ка́зъ, ка́кѡ глаго́лати пѣ̑сни и҆ канѡ́ны, во ст҃ꙋ́ю и҆  вели́кꙋю четыредесѧ́тницꙋ, \hPageref{kakofty}. 

Пѣ̑сни проро́чєскїѧ, \hPageref{proroch}.

Тропарѝ воскре́сны, пое́мїи по непоро́чнахъ, \hPageref{tropneporoch}. 

Прокі́мны воскре́сны ᲂу҆́треннїи на и҃ гласѡ́въ, \hPageref{prochimensunday}.

Тропа́рь глаго́лемый въ недѣ̑ли: Воскрⷭ҇нїе хрⷭ҇то́во  ви́дѣвше: \hPageref{videvshe}. 

Тропа́рь глаго́лемый по н҃_мъ ѱалмѣ̀, є҆гда̀ є҆́сть  полѷеле́й: Мл҃твами а҆пⷭ҇лѡвъ. \hPageref{fiftypsalom}. 

Тропарѝ пое́мїи ѿ недѣ́ли, ꙗ҆́же ѡ҆ мытарѝ, во всю̀  вели́кꙋю четыредесѧ́тницꙋ въ недѣ́лѧхъ: Покаѧ́нїѧ ѿве́рзи мѝ  двє́ри: \hPageref{mytar}. 

По н҃_мъ ѱалмѣ̀ стїхѝ на гдⷭ҇скїѧ и҆ бг҃оро́дичны пра́здники,  пое́мїи вмѣ́стѡ, Мл҃твами: \hPageref{fiftybogorod}.

Всѧ́кое дыха́нїе, пое́мое въ недѣ̑ли и҆ пра́здники, на Хвали́те  гдⷭ҇а съ нб҃съ: \hPageref{vsakoe}.

Пребл҃гослове́нна є҆сѝ бцⷣе дв҃о: \hPageref{preblago}.

Славосло́вїе вели́кое. \hPageref{slavo}

Тропарѝ пое́мїи по непоро́чнахъ ѡ҆ ᲂу҆со́пшихъ \hPageref{neporochsat}. 

Слꙋ́жба на ст҃ꙋ́ю па́схꙋ, \hPageref{pascha}.

Полѷеле́и, пое́мый въ пра́здники гдⷭ҇скїѧ, и҆ бг҃оро́дичны, и҆  наро́читыхъ ст҃ы́хъ, и҆ велича̑нїѧ со и҆збра́нными ѱалмѝ, \hPageref{poly}. 

Припѣ́вы на ѳ҃_й пѣ́сни, на гдⷭ҇скїѧ пра́здники, и҆  бг҃оро́дичны, \hPageref{bogosun}.

%%Ѿпꙋсти́тельнїи тропарѝ воскрⷭ҇ны ѻ҆смѝ гласѡ́въ, \hPageref{богородичны воскр}.

%%Бг҃оро́дичны ѿпꙋсти́тельнїи, по тропарѣ́хъ ст҃ы́хъ, поє́мыѧ  во всѐ лѣ́то, \hPageref{богородичны воскр}. 

\vfil
\centergraphics[width=3cm]{p-006-tocend.pdf}
\vfil
