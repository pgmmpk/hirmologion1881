\centergraphics[width=7cm]{p-116-tone5.pdf}
\cuKinovarCenter{{\ponomarexpandedfont Нача́ло і҆рмосѡ́въ є҃_гѡ гла́са.}}

\cuSec{Пѣ́снь а҃.}{Гла́съ є҃.}
\label{tone.5}

%\vskip-0.6em
\cuLettrine
Конѧ̀ и҆ вса́дника въ мо́ре чермно́е, сокрꙋша́ѧй бра̑ни  мы́шцею высо́кою, хрⷭ҇то́съ и҆стрѧсѐ, і҆и҃лѧ же сп҃сѐ,  побѣ́днꙋю пѣ́снь пою́ща. 
%
\cuSubSec{Вознесе́нїѧ:}

\hKv Сп҃си́телю бг҃ꙋ, въ мо́ри лю́ди немо́крыми  нога́ми наста́вльшемꙋ, и҆ фараѡ́на со всево́инствомъ  пото́пльшемꙋ: томꙋ̀ є҆ди́номꙋ пои́мъ, ꙗ҆́кѡ просла́висѧ. 
%
\cuSubSec{Ѻ҆смогла́сника:}

\hKv Пои́мъ гдⷭ҇ви, сотво́ршемꙋ ди̑внаѧ чꙋдеса̀ въ чермнѣ́мъ  мо́ри, пѣ́снь побѣ́днꙋю, ꙗ҆́кѡ просла́висѧ.  

\hKv Зе́млю, на ню́же не возсїѧ̀, ни ви́дѣ со́лнце когда̀,  бе́зднꙋ, ю҆́же не ви́дѣ на́гꙋ широта̀ нбⷭ҇наѧ, і҆и҃ль  про́йде невла́жнѡ гдⷭ҇и, и҆ вве́лъ є҆сѝ є҆го̀ въ го́рꙋ  ст҃ы́ни твоеѧ̀, хва́лѧща и҆ пою́ща побѣ́днꙋю пѣ́снь. 

\hKv Пѣ́снь побѣ́днꙋю принесе́мъ лю́дїе, и҆зба́вльшемꙋ бг҃ꙋ  и҆з̾ рабо́ты фараѡ́новы мѡѷсє́йскїѧ лю́ди: сла́внѡ бо  просла́висѧ.

\hKv Пои́мъ сп҃си́телю всѣ́хъ, во́льнымъ хотѣ́нїемъ на дре́вѣ  пригвозди́вшемꙋсѧ, и҆ мі́ръ просвѣти́вшемꙋ. 

\hKv Хрⷭ҇то́съ ꙗ҆ви́сѧ на землѝ и҆зба́вивый ро́дъ на́шъ ѿ  і҆дѡлонеи́стовства: томꙋ̀ є҆ди́номꙋ пои́мъ, ꙗ҆́кѡ  просла́висѧ. 

\hKv Пои́мъ гдⷭ҇ви ди́вномꙋ, ꙗ҆́кѡ ѿ го́рькїѧ свободи́вый  рабо́ты і҆и҃лѧ, фараѡ́на со всево́инствомъ погрꙋзѝ. 

\hKv Пѣшеше́ствꙋѧй і҆и҃ль непроходи́мꙋю стезю̀ повелѣ́нїемъ  влады́чнимъ, ра́дꙋѧсѧ поѧ́ше: гдⷭ҇ви пои́мъ, сла́внѡ бо  просла́висѧ.  

\hKv Велича́ваго фараѡ́на въ мо́ри потопи́вша со ѻ҆рꙋ́жїемъ и҆  вса́дники, і҆и҃лѧ же пресла́внѡ спа́сша и҆ по сꙋ́хꙋ  прове́дша, пое́мъ хрⷭ҇та̀, ꙗ҆́кѡ просла́висѧ.
%
\cuSec{Пѣ́снь в҃.}{Гла́съ є҃.}

\hKv Вонмѝ нб҃о, и҆ возглаго́лю, и҆ воспою̀ хрⷭ҇та̀ на́съ ра́ди  ѿ дв҃ы воплоти́вшагосѧ, и҆ тлю̀ и҆ сме́рть стрⷭ҇тїю  разори́вшаго. 
%
\cuSubSec{Во вто́рникъ ѕ҃_ѧ седми́цы поста̀:}

\hKv Ви́дите ви́дите, ꙗ҆́кѡ  а҆́зъ є҆́смь бг҃ъ въ пло́ть ѡ҆блекі́йсѧ во́лею свое́ю, да  сп҃сꙋ̀ а҆да́ма, ѿ ле́сти па́дшаго въ престꙋпле́нїе  ѕмі́емъ. 

\hKv Ви́дите ви́дите, ꙗ҆́кѡ а҆́зъ є҆́смь бг҃ъ ва́шъ въ мо́ри  сп҃сы́й лю́ди, и҆ въ пꙋсты́ни препита́вый ѧ҆̀, и҆ человѣ́къ  бы́въ, да сп҃сꙋ̀ ро́дъ человѣ́ческїй. 

\hKv Ви́дите ви́дите, ꙗ҆́кѡ а҆́зъ є҆́смь бг҃ъ, ѡ҆б̾ ѻ҆́нъ  по́лъ і҆ѻрда́на ѡ҆бходѧ́й, ᲂу҆слы́шавъ, ꙗ҆́кѡ  ла́зарь боли́тъ, и҆ ре́къ: ꙗ҆́кѡ не  ᲂу҆мира́етъ, но є҆́сть сїѐ ѡ҆ сла́вѣ мое́й. 
%
\cuSec{Пѣ́снь г҃.}{Гла́съ є҃.}

\hKv Водрꙋзи́вый на ничесо́мже зе́млю повелѣ́нїемъ твои́мъ, и҆  повѣ́сивый неѡдержи́мо тѧготѣ́ющꙋю, на недви́жимѣмъ  хрⷭ҇тѐ ка́мени за́повѣдей твои́хъ, цр҃ковь твою̀  ᲂу҆твердѝ, є҆ди́не бл҃же и҆ чл҃вѣколю́бче. 
%
\cuSubSec{Вознесе́нїѧ:}

\hKv Си́лою крⷭ҇та̀ твоегѡ̀ хрⷭ҇тѐ, ᲂу҆твердѝ  моѐ помышле́нїе, во є҆́же пѣ́ти и҆ сла́вити сп҃си́тельное  твоѐ вознесе́нїе. 
%
\cuSubSec{Ѻ҆смогла́сника:}

\hKv Оу҆тверди́ ны бж҃е твое́ю си́лою, и҆ низложѝ  є҆реті́чєскаѧ шата̑нїѧ, и҆ вознесѝ ро́гъ на́шъ. 

\hKv Над̾ ꙗ҆зы́ки бг҃ъ воцр҃и́сѧ, бг҃ъ сѣди́тъ на прⷭ҇то́лѣ  ст҃ѣ́мъ свое́мъ, и҆ пое́мъ є҆мꙋ̀ разꙋ́мнѡ, ꙗ҆́кѡ цр҃ю̀ и҆  бг҃ꙋ. 

\hKv Оу҆тверди́ мѧ гдⷭ҇и, въ житїѝ ѡ҆бꙋрева́ема, и҆ рꙋ́кꙋ мнѣ̀  прострѝ, ꙗ҆́кѡ всеси́ленъ.  

\hKv Дви́жимое се́рдце моѐ гдⷭ҇и, волна́ми жите́йскими,  ᲂу҆твердѝ, въ приста́нище ти́хое наставлѧ́ѧ ꙗ҆́кѡ бг҃ъ. 

\hKv На ка́мени за́повѣдей твои́хъ коле́блющагосѧ мѧ̀  ᲂу҆твердѝ, вознесѝ ро́гъ мо́й въ ра́зꙋмѣ повелѣ́нїй  твои́хъ, да хвалѧ́сѧ вопїю̀ тебѣ̀: нѣ́сть ст҃ъ па́че тебѐ  гдⷭ҇и си́лъ. 

\hKv Нѣ́сть ст҃ъ ꙗ҆́кѡ гдⷭ҇ь, и҆ нѣ́сть пра́веденъ ꙗ҆́кѡ бг҃ъ  мо́й, вознесы́й ро́гъ на́шъ, и҆ даѧ́й крѣ́пость царє́мъ  на́шымъ. 

\hKv Оу҆тверди́ мѧ є҆ди́не ще́дре, си́лою крⷭ҇та̀ твоегѡ̀,  и҆́мъ бо хвалю́сѧ: ꙗ҆́кѡ нѣ́сть ст҃ъ, па́че тебѐ гдⷭ҇и. 

\hKv Оу҆твержде́нїе моѐ бꙋ́ди, и҆ прибѣ́жище сп҃се: и҆  црⷭ҇твїѧ твоегѡ̀ сподо́би сло́ве, и҆́стиннымъ се́рдцемъ  и҆сповѣ́дающыѧ влⷣко бж҃е́ственное твоѐ воплоще́нїе. 

\hKv Оу҆твержде́нїе притека́ющихъ къ тебѣ̀ хрⷭ҇тѐ бж҃е, мою̀  мы́сль ᲂу҆твердѝ твои́мъ чл҃вѣколю́бїемъ, и҆ къ позна́нїю  возведѝ за́повѣдей твои́хъ всеси́льне.  

\hKv Оу҆тверди́ ны бж҃е сп҃се твое́ю си́лою, и҆ воздви́гни ро́гъ  твоеѧ̀ цр҃кве, правовѣ́рнѡ хва́лѧщихъ тѧ̀. 

\hKv Держа́вною рꙋко́ю, и҆ си́льнымъ сло́вомъ, нб҃о и҆ зе́млю  сотвори́лъ є҆сѝ, ю҆́же твое́ю кро́вїю и҆скꙋпи́лъ є҆сѝ  цр҃ковь твою̀, ꙗ҆́же ѡ҆ тебѣ̀ ᲂу҆твержда́етсѧ зовꙋ́щи:  ꙗ҆́кѡ нѣ́сть ст҃ъ, ра́звѣ тебѐ гдⷭ҇и. 
%
\cuSec{Пѣ́снь д҃.}{Гла́съ є҃.}

\hKv Бж҃е́ственное твоѐ разꙋмѣ́въ и҆стоща́нїе прозорли́вѡ  а҆ввакꙋ́мъ, хрⷭ҇тѐ, со тре́петомъ вопїѧ́ше тебѣ̀: во  сп҃се́нїе люде́й твои́хъ, сп҃стѝ пома̑занныѧ твоѧ̑ прише́лъ  є҆сѝ. 
%
\cuSubSec{Вознесе́нїѧ:}

\hKv Оу҆слы́шахъ слꙋ́хъ си́лы крⷭ҇та̀, ꙗ҆́кѡ  ра́й ѿве́рзесѧ и҆́мъ, и҆ возопи́хъ сла́ва си́лѣ твое́й  гдⷭ҇и. 
%
\cuSubSec{Ѻ҆смогла́сника:}

\hKv Дѣла̀ смотре́нїѧ твоегѡ̀ гдⷭ҇и, ᲂу҆жаси́ша прⷪ҇ро́ка  а҆ввакꙋ́ма: и҆зше́лъ бо є҆сѝ на сп҃се́нїе  люде́й твои́хъ, сп҃стѝ пома̑занныѧ твоѧ̑ прише́лъ  є҆сѝ. 

\hKv Оу҆слы́шахъ гдⷭ҇и слꙋ́хъ тво́й, и҆ ᲂу҆боѧ́хсѧ: разꙋмѣ́хъ  смотре́нїе твоѐ, и҆ просла́вихъ тѧ̀ є҆ди́не  чл҃вѣколю́бче. 

\hKv Оу҆слы́шахъ слꙋ́хъ си́лы гдⷭ҇ни, и҆ просла́вихъ  чл҃вѣколю́бче, непостижи́мꙋю твою̀ си́лꙋ. 

\hKv Оу҆слы́шахъ гдⷭ҇и, и҆з̾ гро́ба твоѐ воста́нїе, и҆  просла́вихъ твою̀ непобѣди́мꙋю си́лꙋ. 

\hKv Бж҃е́ственнꙋю ꙗ҆вле́нїѧ твоегѡ̀ добро́тꙋ прⷪ҇ро́къ  прови́дѧ, и҆ снизхожде́нїю ᲂу҆жаса́ѧсѧ воплоще́нїѧ хрⷭ҇тѐ,  предвозгласѝ: ѿ ю҆́га прїи́детъ гдⷭ҇ь ѡ҆чище́нїе мі́рови  да́рꙋѧй.  

\hKv Ѿ горы̀ ча́щныѧ прише́ствїе твоѐ ᲂу҆слы́шавъ прⷪ҇ро́къ,  вопїѧ́ше: сла́ва неизрече́нномꙋ воплоще́нїю твоемꙋ̀. 

\hKv Прⷪ҇ро́къ а҆ввакꙋ́мъ прови́дѧ тѧ̀ хрⷭ҇тѐ, бл҃года́рнымъ  гла́сомъ воспроповѣ́даше вопїѧ̀: слꙋ́хъ тво́й ᲂу҆слы́шахъ,  и҆ ᲂу҆боѧ́хсѧ: дѣла̀ твоѧ̑ всѧ̑ разꙋмѣ́хъ и҆ ᲂу҆жасо́хсѧ  гдⷭ҇и.  

\hKv Смотре́нїѧ твоегѡ̀ та́инство разꙋмѣ́въ а҆ввакꙋ́мъ: та́йнѡ  схожде́нїе прише́ствїѧ твоегѡ̀ написꙋ́ѧй, взыва́ше. внегда̀  прїи́детъ вре́мѧ гдⷭ҇и, ꙗ҆ви́шисѧ. 

\hKv Оу҆слы́шахъ слꙋ́хъ тво́й, и҆ ᲂу҆боѧ́хсѧ: разꙋмѣ́хъ дѣла̀  твоѧ̑, и҆ диви́хсѧ гдⷭ҇и. 

\hKv Дх҃омъ бж҃їимъ ѡ҆чи́щьсѧ прⷪ҇ро́къ дыха́ющимъ въ не́мъ,  а҆ввакꙋ́мъ бж҃е́ственный, боѧ́сѧ глаго́лаше: внегда̀  прибли́жатсѧ лѣ̑та, позна́нъ бꙋ́деши бж҃е, на сп҃се́нїе  человѣ́кѡвъ. 
%
\cuSec{Пѣ́снь є҃.}{Гла́съ є҃.}

\hKv Ѡ҆дѣѧ́йсѧ свѣ́томъ ꙗ҆́кѡ ри́зою, къ тебѣ̀ ᲂу҆́тренюю,  и҆ тебѣ̀ зовꙋ̀: дꙋ́шꙋ мою̀ просвѣтѝ ѡ҆мраче́ннꙋю хрⷭ҇тѐ,  ꙗ҆́кѡ є҆ди́нъ бл҃гоꙋтро́бенъ. 
%
\cuSubSec{Вознесе́нїѧ:}

\hKv Оу҆́тренююще вопїе́мъ тѝ гдⷭ҇и, сп҃си́ ны:  ты́ бо є҆сѝ бг҃ъ на́шъ, ра́звѣ бо тебѐ и҆но́гѡ не  зна́емъ.  
%
\cuSubSec{Ѻ҆смогла́сника:}

\hKv Ѿ но́щи покланѧ́ющыѧсѧ тебѣ̀ хрⷭ҇тѐ, поми́лꙋй, и҆ ми́ръ  да́рꙋй: занѐ свѣ́тъ повелѣ̑нїѧ твоѧ̑, бы́ша и҆сцѣлє́нїѧ  рабѡ́мъ твои̑мъ чл҃вѣколю́бче. 

\hKv Свѣ́те и҆́стинный хрⷭ҇тѐ бж҃е, къ тебѣ̀ ᲂу҆́тренюетъ  дꙋ́хъ мо́й ѿ но́щи, ꙗ҆вѝ на мѧ̀ лицѐ твоѐ. 

\hKv Ѻ҆каѧ́ннꙋю дꙋ́шꙋ мою̀ нощеборю́щꙋюсѧ со тьмо́ю страсте́й,  предвари́въ ᲂу҆ще́дри, и҆ возсїѧ́й мы́сленное сл҃нце  днесвѣ́тлыѧ ѕвѣзды̀ во мнѣ̀, во є҆́же раздѣли́ти но́щь  ѿ свѣ́та. 

\hKv Пра́вды сл҃нце ᲂу҆́мное просвѣти́ мѧ, но́щїю страсте́й  содержи́маго, и҆ наста́ви къ бж҃е́ственнѣй стезѝ, ꙗ҆́кѡ  є҆ди́нъ є҆сѝ ми́ра цр҃ь. 

\hKv Не паде́мъ, но падꙋ́тъ въ паде́нїе, ѡ҆́браза твоегѡ̀ бцⷣе  ᲂу҆кори́телїе, да во́змꙋтсѧ ѿ среды̀ на́съ, да не ви́дѧтъ  сла́вꙋ твою̀ не разꙋмѣ́ющїи: занѐ свѣ́тъ и҆ ми́ръ твоегѡ̀  сн҃а повелѣ̑нїѧ.  

\hKv Гдⷭ҇и бж҃е на́шъ, стѧжи́ ны, гдⷭ҇и, ра́звѣ тебѐ и҆но́гѡ не  вѣ́мы, и҆́мѧ твоѐ и҆менꙋ́емъ, возсїѧ́й на всѧ̑ свѣ́тъ. 

\hKv Пра́вдѣ наꙋчи́тесѧ живꙋ́щїи на землѝ, и҆ ᲂу҆́мъ  ѡ҆чи́стимъ, воздви́женнымъ се́рдцемъ къ ще́дромꙋ бг҃ꙋ ѿ  но́щи ᲂу҆́тренююще. 

\hKv Свѣ́та непристꙋ́пнаго хрⷭ҇тѐ живота̀ вѣ́чнаго, ѿ но́щи  ᲂу҆́тренююще, пое́мъ тѧ̀ є҆ди́наго свѣтода́вца бг҃а. 

\hKv Возсїѧ́й хрⷭ҇тѐ бж҃е, свѣ́тъ ра́зꙋма твоегѡ̀, и҆  просвѣтѝ сердца̀ на̑ша, и҆ сп҃сѝ дꙋ́шы на́шѧ. 

\hKv Сло́во бж҃їе всеси́льное, ми́ръ всемꙋ̀ мі́рꙋ послѝ. и҆  свѣ́томъ и҆́стиннымъ ѡ҆свѣща́ѧ и҆ просвѣща́ѧ всѧ̑, и҆з̾  но́щи тѧ̀ сла́вѧщыѧ. 
%
\cuSec{Пѣ́снь ѕ҃.}{Гла́съ є҃.}

\hKv Неи́стовствꙋющеесѧ бꙋ́рею дꙋшетлѣ́нною, влⷣко хрⷭ҇тѐ,  страсте́й мо́ре ᲂу҆кротѝ, и҆ ѿ тлѝ возведи́ мѧ, ꙗ҆́кѡ  бл҃гоꙋтро́бенъ.  
%
\cuSubSec{Вознесе́нїѧ:}

\hKv Ѡ҆бы́де мѧ̀ бе́здна, гро́бъ мнѣ̀ ки́тъ  бы́сть: а҆́зъ же возопи́хъ къ тебѣ̀ чл҃вѣколю́бцꙋ, и҆  сп҃се́ мѧ десни́ца твоѧ̀ гдⷭ҇и. 
%
\cuSubSec{Ѻ҆смогла́сника:}

\hKv Ѿ ки́та прⷪ҇ро́ка и҆зба́вилъ є҆сѝ, мене́ же и҆з̾  глꙋбины̀ грѣхѡ́въ возведѝ гдⷭ҇и, и҆ сп҃си́ мѧ. 

\hKv Ꙗ҆́коже прⷪ҇ро́ка ѿ ѕвѣ́рѧ и҆зба́вилъ є҆сѝ гдⷭ҇и, и҆  менѐ и҆з̾ глꙋбины̀ несодержи́мыхъ страсте́й возведѝ,  молю́сѧ, да приложꙋ̀ призрѣ́ти мѝ ко хра́мꙋ ст҃о́мꙋ  твоемꙋ̀. 

\hKv Бꙋ́рѧ содержа́ мѧ сп҃се грѣхо́внаѧ, и҆ дꙋ́шꙋ мою̀  низво́дитъ въ дꙋшетли́тельнꙋю глꙋбинꙋ̀: но ꙗ҆́кѡ дре́вле  і҆ѡ́нꙋ хрⷭ҇тѐ возведѝ, ꙗ҆́кѡ да пожрꙋ́ ти и҆ а҆́зъ со  гла́сомъ хвале́нїѧ. 

\hKv Ѿ чре́ва а҆́дова, во́плѧ моегѡ̀ ᲂу҆слы́шалъ є҆сѝ гла́съ  мо́й, и҆ и҆зба́вилъ є҆сѝ ѿ тлѝ живо́тъ мо́й,  многомлⷭ҇тиве.  

\hKv Возопѝ ѿ послѣ́днихъ а҆да́мъ: бж҃е мо́й, и҆зба́ви мѧ̀  па́дшаго, и҆ ᲂу҆подо́бивсѧ прїидѝ, во є҆́же сп҃сти́ ны. 

\hKv Во глꙋбины̀ ѿве́ргсѧ морскі̑ѧ, ѿ нꙋ́ждъ мои́хъ и҆зба́ви  мѧ̀, ꙗ҆́коже ѿ ки́та возве́лъ є҆сѝ прⷪ҇ро́ка і҆ѡ́нꙋ,  та́кѡ и҆ менѐ, молю́сѧ, предвари́въ сп҃сѝ ѿ жите́йскихъ  ѕѡ́лъ. 

\hKv Въ ки́тово чре́во подше́дый і҆ѡ́на, не растлѣ́сѧ влⷣко,  твои́мъ повелѣ́нїемъ, но проѡбраже́нїѧ бы́сть тридне́внагѡ  твоегѡ̀ погребе́нїѧ. тѣ́мъ тѝ взыва́ше: да взы́детъ ѿ  тлѝ живо́тъ мо́й къ тебѣ̀ гдⷭ҇и. 

\hKv Возопи́хъ всѣ́мъ се́рдцемъ мои́мъ къ ще́дромꙋ бг҃ꙋ, и҆  ᲂу҆слы́ша мѧ̀ ѿ а҆́да преиспо́днѧгѡ, и҆ возведѐ ѿ тлѝ  живо́тъ мо́й.  

\hKv Возопи́хъ въ печа́ли мое́й, сп҃се мо́й, молю́сѧ, и҆зба́ви  мѧ̀ ѿ ѡ҆бдержа́щїѧ мѧ̀ болѣ́зни и҆ ѡ҆ѕлобле́нїѧ, ꙗ҆́кѡ  млⷭ҇тивъ помѧнꙋ́въ твою̀ млⷭ҇ть. 

\hKv Въ цр҃ковь нбⷭ҇нꙋю ст҃ꙋ́ю твою̀, да прїи́детъ  мл҃тва моѧ̀, вопїю́ ти ꙗ҆́кѡ і҆ѡ́на,  и҆з̾ глꙋбины̀ се́рдца морска́гѡ: ѿ грѣ̑хъ мои́хъ  возведи́ мѧ, молю́сѧ тебѣ̀ гдⷭ҇и. 
%
\cuSec{Пѣ́снь з҃.}{Гла́съ є҃.}

\hKv Превозноси́мый ѻ҆тцє́въ гдⷭ҇ь, пла́мень ᲂу҆гасѝ,  ѻ҆́троки ѡ҆росѝ, согла́снѡ пою́щыѧ: бж҃е бл҃гослове́нъ  є҆сѝ. 
%
\cuSubSec{Вознесе́нїѧ:}

\hKv Въ пещѝ ѻ҆́гненнѣй пѣсносло́вцы сп҃сы́й  ѻ҆́троки, бл҃гослове́нъ бг҃ъ ѻ҆тє́цъ на́шихъ. 
%
\cuSubSec{Ѻ҆смогла́сника:}

\hKv Ѻ҆гнѧ̀ гаси́лище ѻ҆трокѡ́въ мл҃твы, ѡ҆роша́ющаѧ пе́щь,  проповѣ́дница чꙋдесѐ не ѡ҆палѧ́ющи нижѐ сожига́ющи  пѣсносло́вцы бг҃а ѻ҆тє́цъ на́шихъ. 

\hKv Бл҃гослове́нъ є҆сѝ бж҃е, ви́дѧй бє́здны, и҆ на прⷭ҇то́лѣ  сла́вы сѣдѧ́й препѣ́тый и҆ пресла́вный.  

\hKv Пе́щь ᲂу҆гаси́вый въ вавѷлѡ́нѣ ѻ҆́гненнꙋю, и҆ ѻ҆́троки  въ не́й, ꙗ҆́кѡ въ черто́зѣ, сохрани́вый, бл҃гослове́нъ  є҆сѝ бж҃е ѻ҆тє́цъ на́шихъ. 

\hKv Прехва́льно є҆́сть и҆ препросла́влено и҆́мѧ твоѐ, гдⷭ҇и  бж҃е ѻ҆тє́цъ на́шихъ, во всѧ̑ вѣ́ки: и҆ препросла́вленъ,  ꙗ҆́кѡ пра́веденъ є҆сѝ ѡ҆ всѣ́хъ, ꙗ҆̀же сотвори́лъ  є҆сѝ на́мъ. 

\hKv Ѻ҆́троки въ пещѝ а҆́гг҃ломъ сохрани́вый, провозвѣща́ѧ  и҆́ми ро́ждшꙋю тѧ̀ нетлѣ́ннꙋю дв҃ꙋ: бл҃гослове́нъ є҆сѝ  бж҃е ѻ҆тє́цъ на́шихъ. 

\hKv Ѡ҆роша́емаѧ пе́щь и҆ ꙗ҆́рость мꙋчи́тельнаѧ бл҃гочести̑выѧ  не ᲂу҆страшѝ ѻ҆́троки, де́рзостнѡ вопїю́щыѧ:  бл҃гослове́нъ є҆сѝ бж҃е ѻ҆тє́цъ на́шихъ. 

\hKv А҆́гг҃ломъ ѻ҆́троки сп҃сы́й, и҆ гремѧ́щꙋю пе́щь преложи́вый  въ ро́сꙋ, бл҃гослове́нъ є҆сѝ гдⷭ҇и. 

\hKv Трⷪ҇цы сла́вы бг҃очести́вїи и҆з̾ѡбрази́вше ѻ҆́троцы, пе́щь  дре́вле ѻ҆гне́мъ пали́мꙋю въ ро́сꙋ преложи́ша, и҆ пою́ще  поѧ́хꙋ: бл҃гослове́нъ є҆сѝ бж҃е ѻ҆тє́цъ на́шихъ.  

\hKv Пла́мень пе́щный порабо́тиша бл҃гочести́вїи ѻ҆́троцы,  тѣ́мъ свы́ше ѡ҆росѝ ѧ҆̀ гото́вы пали́мы по є҆стествꙋ̀, но  па́че є҆стества̀ мꙋ́жески поѧ́хꙋ: бл҃гослове́нъ є҆сѝ  гдⷭ҇и, на прⷭ҇то́лѣ сла́вы црⷭ҇твїѧ твоегѡ̀. 
%
\cuSec{Пѣ́снь и҃.}{Гла́съ є҃.}

\hKv Тебѣ̀ вседѣ́телю, въ пещѝ ѻ҆́троцы въ мі́рный ли́къ  спле́тше поѧ́хꙋ: дѣла̀ всѧ̑каѧ гдⷭ҇а по́йте, и҆  превозноси́те во всѧ̑ вѣ́ки. 
%
\cuSubSec{Вознесе́нїѧ:}

\hKv И҆з̾ ѻ҆ц҃а̀ пре́жде вѣ̑къ рожде́ннаго сн҃а  и҆ бг҃а, и҆ въ послѣ̑днѧѧ лѣ̑та воплоще́ннаго ѿ дв҃ы  мт҃ре, сщ҃е́нницы по́йте, лю́дїе превозноси́те во всѧ̑  вѣ́ки. 
%
\cuSubSec{Ѻ҆смогла́сника:}

\hKv Творца̀ тва́ри, є҆гѡ́же ᲂу҆жаса́ютсѧ а҆́гг҃ли, по́йте  лю́дїе, и҆ превозноси́те во всѧ̑ вѣ́ки. 

\hKv А҆́гг҃лѡвъ со́нмъ, человѣ́кѡвъ собо́ръ, цр҃ѧ̀ и҆ зижди́телѧ  всѣ́хъ, сщ҃е́нницы по́йте,   бл҃гослови́те леві́ти, лю́дїе превозноси́те во всѧ̑ вѣ́ки. 

\hKv Всѣ́хъ содѣ́телѧ, и҆ сп҃са дꙋ́шъ на́шихъ, сщ҃е́нницы  бл҃гослови́те, лю́дїе превозноси́те є҆го̀ во вѣ́ки. 

\hKv Прпⷣбнїи твоѝ ѻ҆́троцы въ пещѝ хрⷭ҇тѐ воспѣва́юще  глаго́лахꙋ: бл҃гослови́те всѧ̑ дѣла̀ гдⷭ҇нѧ гдⷭ҇а. 

\hKv Ѻ҆те́ческое велѣ́нїе мꙋ́дрствꙋюще ю҆́нѡши трѝ въ  вавѷлѡ́нѣ, безбо́жномꙋ велѣ́нїю бл҃гоче́стнѡ не  покарѧ́ющесѧ, посредѣ̀ ѡ҆сꙋжде́нїѧ ѻ҆́гненнагѡ стоѧ́ще  вопїѧ́хꙋ: бл҃гослови́те всѧ̑ дѣла̀ гдⷭ҇нѧ гдⷭ҇а. 

\hKv Зижди́телю всѣ́хъ бг҃оно́снїи ѻ҆́троцы въ пещѝ пѣ́снь  приноша́хꙋ, и҆ пою́ще вопїѧ́хꙋ: бл҃гослови́те всѧ̑ дѣла̀  гдⷭ҇нѧ гдⷭ҇а. 

\hKv Ѡ҆роси́вшаго пе́щь, и҆ ѻ҆́троки сохра́ньшаго посредѣ̀  горѧ́щагѡ пла́мене, ѻ҆́троцы славосло́вите, сщ҃е́нницы  бл҃гослови́те, лю́дїе превозноси́те во всѧ̑ вѣ́ки.  

\hKv Ѿ ѻ҆ц҃а̀ рожде́ннаго пре́жде вѣ̑къ бг҃а сло́ва, по́йте  сщ҃е́нницы, лю́дїе превозноси́те во всѧ̑ вѣ́ки. 

\hKv Цр҃ѧ̀ хрⷭ҇та̀, є҆го́же пою́тъ херꙋві́ми, и҆ сла́вѧтъ  серафі́ми, по́йте лю́дїе, и҆ превозноси́те во всѧ̑ вѣ́ки. 

\hKv Крѣ́пцыи ѻ҆́троцы трїѐ сꙋ́ще, си́лою ѡ҆бо́лкшесѧ ст҃ы́ѧ  трⷪ҇цы, ᲂу҆лови́ша и҆ побѣди́ша халдє́и, и҆ ди́внѡ  и҆змѣни́сѧ є҆стество̀: ка́кѡ ѻ҆́гнь въ ро́сꙋ  прелага́шесѧ; без̾ тꙋ́ги ѧ҆̀ сохранѝ ꙗ҆́кѡ пелена́ми. ѽ  пролїѧ́вый мꙋ́дрость на всѣ́хъ дѣ́лѣхъ твои́хъ бж҃е, тѧ̀  превозно́симъ во вѣ́ки. 
%
\cuSec{Пѣ́снь ѳ҃.}{Гла́съ є҃.}

\hKv И҆са́їе ликꙋ́й, дв҃а и҆мѣ̀ во чре́вѣ, и҆ родѝ сн҃а  є҆мманꙋ́ила, бг҃а же и҆ чл҃вѣ́ка, восто́къ и҆́мѧ є҆мꙋ̀:  є҆го́же велича́юще дв҃ꙋ ᲂу҆бл҃жа́емъ.  
%
\cuSubSec{Вознесе́нїѧ:}

\hKv Тѧ̀ па́че ᲂу҆ма̀ и҆ словесѐ мт҃рь бж҃їю, въ  лѣ́то безлѣ́тнаго неизрече́ннѡ ро́ждшꙋю, вѣ́рнїи  є҆диномꙋ́дреннѡ велича́емъ. 
%
\cuSubSec{Ѻ҆смогла́сника:}

\hKv Тѧ̀ бл҃же́ннꙋю въ жена́хъ, и҆ бл҃гослове́ннꙋю бг҃омъ,  человѣ́ческїй ро́дъ пѣ́сньми велича́емъ. 

\hKv Ꙗ҆́кѡ сотворѝ тебѣ̀ вели̑чїѧ си́льный, дв҃ꙋ ꙗ҆́вль тѧ̀  чи́стꙋ по ржⷭ҇твѣ̀, ꙗ҆́кѡ ро́ждшꙋю без̾ сѣ́мене своего̀  творца̀: тѣ́мъ тѧ̀ бцⷣе велича́емъ. 

\hKv Велича́емъ хрⷭ҇тѐ, твою̀ всенепоро́чнꙋю мт҃рь чⷭ҇тꙋю ꙗ҆́кѡ  роди́ тѧ преесте́ственнѡ пло́тїю, ѿ всѧ́кїѧ льстѝ и҆ тлѝ  на́съ и҆збавлѧ́ѧй. 

\hKv Чре́во твоѐ ꙗ҆ви́сѧ ши́ршее нб҃съ, и҆ сла́вншаѧ є҆ди́на  ꙗ҆ви́ласѧ є҆сѝ, чꙋде́съ превы́шшаѧ є҆ди́на показа́ласѧ  є҆сѝ бцⷣе, ро́ждшаѧ невмѣсти́маго и҆ присносꙋ́щнаго  бг҃а: по до́лгꙋ тѧ̀ человѣ́ческїй ро́дъ непреста́ннѡ  велича́емъ.  

\hKv Ѽ чꙋ́до! и҆́бо чꙋ́до соверша́емое въ тебѣ̀ бцⷣе:  ѡ҆бновлѧ́етсѧ є҆стество̀, и҆ бг҃ъ чл҃вѣ́къ быва́етъ, и҆ во  двꙋ̀ є҆стєствꙋ̀ бг҃осло́вѧще велича́емъ. 

\hKv Тѧ̀ бж҃їю мт҃рь, и҆ дв҃ꙋ чⷭ҇тꙋю, и҆ херꙋві́мѡвъ сла́вншꙋю,  во гла́сѣхъ пѣ́сней велича́емъ. 

\hKv Же́злъ прозѧ́бшїй ѿ ко́рене дв҃дова, бцⷣе, всепѣ́таѧ,  цвѣ́тъ краснѣ́йшїй на́мъ возсїѧ́ла є҆сѝ, дре́внѧгѡ  вино́вный бл҃же́нства. тѣ́мъ тѧ̀ всѝ пѣ́сньми велича́емъ. 

\hKv Тѧ̀ рабꙋ̀ и҆ мт҃рь хрⷭ҇та̀ бг҃а на́шегѡ бцⷣе, досто́йнѡ  велича́емъ. 

\hKv Тѧ̀ неизрече́ннымъ сло́вомъ созда́телѧ заче́ншꙋю во чре́вѣ  велича́емъ, ра́дость бо ро́ждшаѧ живо́тъ всѣ̑мъ дарова́ла  є҆сѝ, всебл҃же́ннаѧ. 

\hKv Тѧ̀ неискꙋсомꙋ́жнꙋю невѣ́стꙋ, мт҃рь сло́ва, ст҃ꙋ́ю дв҃ꙋ  велича́емъ. 

\hKv И҆з̾ є҆де́ма и҆зы́де ро́дъ на́шъ пра́бабы ра́ди є҆́ѵы:  при́званъ же тобо́ю ро́ждшею на́мъ  но́ваго а҆да́ма хрⷭ҇та̀, є҆стєствꙋ̀ дв҃о чⷭ҇таѧ.  взыгра́сѧ а҆да́мъ пра́дѣдъ, ꙗ҆́кѡ и҆збы́въ пе́рвыѧ  клѧ́твы: мы́ же тобо́ю хва́лѧщесѧ, ꙗ҆́кѡ тебѐ ра́ди бг҃а  позна́хомъ, и҆ тебѐ велича́емъ.  
