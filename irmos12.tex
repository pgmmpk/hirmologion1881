\centergraphics[width=7cm]{p-197-irmos2.pdf}
\hKv Прокі́мны воскрⷭ҇ны. 

\hKv Прокі́менъ, гла́съ а҃: Ны́нѣ воскрⷭ҇нꙋ̀, глаго́летъ гдⷭ҇ь,  положꙋ́сѧ во сп҃се́нїе, не ѡ҆биню́сѧ ѡ҆ не́мъ. Сті́хъ:  Словеса̀ гдⷭ҇нѧ, словеса̀ чи́ста. 

\hKv Прокі́менъ, гла́съ в҃: Воста́ни гдⷭ҇и бж҃е мо́й  повелѣ́нїемъ, и҆́мже заповѣ́далъ є҆сѝ, и҆ со́нмъ люде́й  ѡ҆бы́детъ тѧ̀. Сті́хъ: Гдⷭ҇и бж҃е мо́й, на тѧ̀ ᲂу҆пова́хъ,  сп҃си́ мѧ. 

\hKv Прокі́менъ, гла́съ г҃: Рцы́те во ꙗ҆зы́цѣхъ, ꙗ҆́кѡ гдⷭ҇ь  воцр҃и́сѧ: и҆́бо и҆спра́ви вселе́ннꙋю, ꙗ҆́же не  подви́житсѧ. Сті́хъ: Воспо́йте гдⷭ҇еви пѣ́снь но́вꙋ. 

\hKv Прокі́менъ, гла́съ д҃: Воскрⷭ҇нѝ гдⷭ҇и, помозѝ на́мъ, и҆  и҆зба́ви на́съ и҆́мене твоегѡ̀ ра́ди. Сті́хъ: Бж҃е,  ᲂу҆ши́ма на́шима ᲂу҆слы́шахомъ.  

\hKv Прокі́менъ, гла́съ є҃: Воскрⷭ҇нѝ гдⷭ҇и бж҃е мо́й, да  вознесе́тсѧ рꙋка̀ твоѧ̀, ꙗ҆́кѡ ты̀ црⷭ҇твꙋеши во вѣ́ки.  Сті́хъ: И҆сповѣ́мсѧ тебѣ̀ гдⷭ҇и, всѣ́мъ се́рдцемъ мои́мъ. 

\hKv Прокі́менъ, гла́съ ѕ҃: Гдⷭ҇и, воздви́гни си́лꙋ твою̀, и҆  прїидѝ во є҆́же сп҃стѝ на́съ. Сті́хъ: Пасы́й і҆и҃лѧ  вонмѝ, наставлѧ́ѧй ꙗ҆́кѡ ѻ҆вча̀ і҆ѡ́сифа. 

\hKv Прокі́менъ, гла́съ з҃: Воскрⷭ҇нѝ гдⷭ҇и бж҃е мо́й, да  вознесе́тсѧ рꙋка̀ твоѧ̀, не забꙋ́ди ᲂу҆бо́гихъ твои́хъ до  конца̀. Сті́хъ: И҆сповѣ́мсѧ тебѣ̀, гдⷭ҇и, всѣ́мъ  се́рдцемъ мои́мъ.  

\hKv Прокі́менъ, гла́съ и҃: Воцр҃и́тсѧ гдⷭ҇ь во вѣ́къ, бг҃ъ  тво́й, сїѡ́не, въ ро́дъ и҆ ро́дъ. Сті́хъ: Хвалѝ дꙋшѐ моѧ̀  гдⷭ҇а, восхвалю̀ гдⷭ҇а въ животѣ̀ мое́мъ. 

\hKv По є҆ѵⷢ҇лїи, а҆́ще є҆́сть недѣ́лѧ, глаго́лемъ се́й  тропа́рь: 

\hKv Воскрⷭ҇нїе хрⷭ҇то́во ви́дѣвше, поклони́мсѧ ст҃о́мꙋ гдⷭ҇ꙋ  і҆и҃сꙋ, є҆ди́номꙋ безгрѣ́шномꙋ,  крⷭ҇тꙋ̀ твоемꙋ̀ покланѧ́емсѧ, хрⷭ҇тѐ, и҆ ст҃о́е  воскрⷭ҇нїе твоѐ пое́мъ и҆ сла́вимъ: ты́ бо є҆сѝ бг҃ъ  на́шъ, ра́звѣ тебѣ̀ и҆но́гѡ не зна́емъ, и҆́мѧ твоѐ  и҆менꙋ́емъ. прїиди́те всѝ вѣ́рнїи, поклони́мсѧ ст҃о́мꙋ  хрⷭ҇то́вꙋ воскрⷭ҇нїю: се́ бо прїи́де крⷭ҇то́мъ ра́дость  всемꙋ̀ мі́рꙋ. всегда̀ бл҃гословѧ́ще гдⷭ҇а, пое́мъ воскрⷭ҇нїе  є҆гѡ̀: распѧ́тїе бо претерпѣ́въ, сме́ртїю сме́рть  разрꙋшѝ. 

\hKv По н҃_мъ ѱалмѣ̀: 

\hKv Сла́ва: Мл҃твами а҆пⷭ҇лѡвъ, млⷭ҇тиве, ѡ҆чи́сти мно́жєства  согрѣше́нїй на́шихъ.  

\hKv И҆ ны́нѣ: Мл҃твами бцⷣы, млⷭ҇тиве, ѡ҆чи́сти мно́жєства  согрѣше́нїй на́шихъ. 

\hKv Та́же, гла́съ ѕ҃: 

\hKv Поми́лꙋй мѧ̀, бж҃е, по вели́цѣй млⷭ҇ти твое́й, и҆ по  мно́жествꙋ щедро́тъ твои́хъ, ѡ҆чи́сти беззако́нїе моѐ.  

\hKv Та́же: Воскр҃съ і҆и҃съ ѿ гро́ба, ꙗ҆́коже проречѐ, дадѐ  на́мъ живо́тъ вѣ́чный, и҆ ве́лїю млⷭ҇ть. 

\hKv А҆́ще ли недѣ́лѧ мытарѧ̀, и҆лѝ блꙋ́днагѡ, и҆лѝ  мѧсопꙋ́стнаѧ, и҆ сыропꙋ́стнаѧ, и҆ въ про́чыѧ недѣ̑ли  вели́кагѡ поста̀, 

\hKv вмѣ́стѡ, Мл҃твами а҆пⷭ҇лъ: 

\hKv Сла́ва, гла́съ и҃: Покаѧ́нїѧ ѿве́рзи мѝ двє́ри  жизнода́вче: ᲂу҆́тренюетъ бо дх҃ъ мо́й ко хра́мꙋ ст҃о́мꙋ  твоемꙋ̀, хра́мъ носѧ́й тѣле́сный ве́сь ѡ҆скверне́нъ: но  ꙗ҆́кѡ ще́дръ ѡ҆чи́сти бл҃гоꙋтро́бною твое́ю млⷭ҇тїю. 

\hKv И҆ ны́нѣ, бг҃оро́диченъ: На сп҃се́нїѧ стєзѝ наста́ви мѧ̀,  бцⷣе, стꙋ́дными бо ѡ҆калѧ́хъ дꙋ́шꙋ грѣхмѝ, и҆ въ  лѣ́ности всѐ житїѐ моѐ и҆жди́хъ: но твои́ми мл҃твами  и҆зба́ви мѧ̀ ѿ всѧ́кїѧ нечистоты̀. 

\hKv Та́же, гла́съ ѕ҃:  

\hKv Поми́лꙋй мѧ̀ бж҃е: 
%
\cuSubSec{пи́санъ вы́ше сегѡ̀.}

\hKv Посе́мъ: Мно́жєства содѣ́ѧнныхъ мно́ю лю́тыхъ, помышлѧ́ѧ  ѻ҆каѧ́нный, трепе́щꙋ стра́шнагѡ днѐ сꙋ́днагѡ: но надѣ́ѧсѧ  на млⷭ҇ть бл҃гоꙋтро́бїѧ твоегѡ̀, ꙗ҆́кѡ дв҃дъ вопїю́ тѝ:  поми́лꙋй мѧ̀, бж҃е, по вели́цѣй твое́й млⷭ҇ти. 

\hKv На введе́нїе прест҃ы́ѧ бцⷣы, вмѣ́стѡ, Мл҃твами бцⷣы: 

\hKv Сла́ва, гла́съ в҃: Дне́сь хра́мъ ѡ҆дꙋшевле́нный, и҆  вели́кагѡ цр҃ѧ̀, въ хра́мъ вхо́дитъ томꙋ̀ ᲂу҆гото́ватисѧ въ  бж҃е́ственное жили́ще: лю́дїе весели́тесѧ. 

\hKv И҆ ны́нѣ, то́йже. 

\hKv На ржⷭ҇тво̀ хрⷭ҇то́во. 

\hKv Сла́ва: Всѧ́чєскаѧ дне́сь ра́дости и҆сполнѧ́ютсѧ, хрⷭ҇то́съ  роди́сѧ ѿ дв҃ы. 

\hKv И҆ ны́нѣ: Всѧ́чєскаѧ дне́сь ра́дости и҆сполнѧ́ютсѧ,  хрⷭ҇то́съ роди́сѧ въ виѳлее́мѣ.  

\hKv На бг҃оѧвле́нїе. 

\hKv Сла́ва: Всѧ́чєскаѧ дне́сь да возра́дꙋютсѧ хрⷭ҇тꙋ̀  ꙗ҆́вльшꙋсѧ во і҆ѻрда́нѣ.  

\hKv И҆ ны́нѣ, то́йже. 

\hKv Въ недѣ́лю ва́їй. 

\hKv Сла́ва, гла́съ в҃: Дне́сь хрⷭ҇то́съ вхо́дитъ во гра́дъ  виѳа́нїю, на жребѧ́ти сѣдѧ́й, безслове́сїе разрѣша́ѧ  ꙗ҆зы̑къ ѕлѣ́йшее, дре́вле свирѣ́пѣющее. 

\hKv И҆ ны́нѣ, па́ки то́йже. 

\hKv На преѡбраже́нїе гдⷭ҇не. 

\hKv Сла́ва: Всѧ́чєскаѧ дне́сь ра́дости и҆спо́лнишасѧ, хрⷭ҇то́съ  преѡбрази́сѧ пред̾ ᲂу҆ч҃нкѝ. 

\hKv И҆ ны́нѣ, то́йже. 

\hKv По канѡ́нѣ, а҆́ще є҆́сть недѣ́лѧ:  

\hKv Ст҃ъ гдⷭ҇ь: и҆ ᲂу҆́треннїй є҆ѯапостїла́рїй. 

\hKv А҆́ще же є҆́сть м҃_ца, глаго́лемъ свѣти́льны въ  прилꙋчи́вшїйсѧ гла́съ. И҆ а҆́бїе хвали́тны.  

\hKv А҆́ще ᲂу҆́бѡ є҆́сть недѣ́лѧ, и҆лѝ влⷣчнїй пра́здникъ  и҆лѝ ст҃ы́й и҆мѣ́ѧй вели́кое славосло́вїе, пое́тсѧ си́це:  Всѧ́кое дыха́нїе, въ прилꙋчи́вшїйсѧ гла́съ. 

\hKv Всѧ́кое дыха́нїе да хва́литъ гдⷭ҇а. Хвали́те гдⷭ҇а съ нб҃съ,  хвали́те є҆го̀ въ вы́шнихъ. Тебѣ̀ подоба́етъ пѣ́снь бг҃ꙋ. 

\hKv А҆́ще же и҆ны́й де́нь, си́це: 

\hKv Пе́рвый ли́къ: Хвали́те гдⷭ҇а съ нб҃съ. Тебѣ̀ подоба́етъ  пѣ́снь бг҃ꙋ. 

\hKv Па́ки пе́рвый ли́къ: Хвали́те гдⷭ҇а съ нб҃съ, хвали́те  є҆го̀ въ вы́шнихъ. Тебѣ̀ подоба́етъ пѣ́снь бг҃ꙋ. 

\hKv Вторы́й ли́къ: Хвали́те є҆го̀ всѝ а҆́гг҃ли є҆гѡ̀,  хвали́те є҆го̀ всѧ̑ си̑лы є҆гѡ̀. Тебѣ̀ подоба́етъ пѣ́снь  бг҃ꙋ. 

\hKv а҃ Хвали́те є҆го̀ со́лнце и҆ лꙋна̀, хвали́те є҆го̀ всѧ̑  ѕвѣ́зды и҆ свѣ́тъ. в҃ Хвали́те є҆го̀ нб҃са̀ нб҃съ, и҆  вода̀, ꙗ҆́же превы́ше нб҃съ, да восхва́лѧтъ и҆́мѧ гдⷭ҇не.  а҃ Ꙗ҆́кѡ то́й речѐ, и҆ бы́ша: то́й повелѣ̀, и҆  созда́шасѧ. в҃ Поста́ви  ѧ҆̀ въ  вѣ́къ, и҆ въ вѣ́къ вѣ́ка: повелѣ́нїе положѝ, и҆ не  ми́мѡ и҆́детъ. а҃ Хвали́те гдⷭ҇а ѿ землѝ ѕмі́евє, и҆  всѧ̑ бє́здны: в҃ Ѻ҆́гнь, гра́дъ, снѣ́гъ, го́лоть, дх҃ъ  бꙋ́ренъ, творѧ̑щаѧ сло́во є҆гѡ̀. а҃ Го́ры и҆ всѝ хо́лми,  древа̀ плодонѡ́сна и҆ всѝ ке́дри. в҃ Ѕвѣ́рїе и҆ всѝ  ско́ти, га́ди и҆ пти̑цы перна́ты. а҃ Ца́рїе зе́мстїи и҆  всѝ лю́дїе, кнѧ́зи и҆ всѝ сꙋдїи̑ зе́мстїи, в҃ Ю҆́нѡши и҆  дѣ̑вы, ста́рцы съ ю҆́нотами, да восхва́лѧтъ и҆́мѧ гдⷭ҇не,  ꙗ҆́кѡ вознесе́сѧ и҆́мѧ тогѡ̀ є҆ди́нагѡ. а҃  И҆сповѣ́данїе є҆гѡ̀ на землѝ и҆ на нб҃сѝ, и҆ вознесе́тъ  ро́гъ люде́й свои́хъ. в҃ Пѣ́снь всѣ̑мъ прпⷣбнымъ  є҆гѡ̀, сыновѡ́мъ і҆и҃лєвымъ, лю́демъ, приближа́ющымсѧ  є҆мꙋ̀. а҃ Воспо́йте гдⷭ҇ви пѣ́снь но́вꙋ, хвале́нїе въ  цр҃кви прпⷣбныхъ. в҃ Да возвесели́тсѧ і҆и҃ль ѡ҆  сотво́ршемъ є҆го̀, и҆ сы́нове сїѡ́ни возра́дꙋютсѧ ѡ҆ цр҃ѣ̀  свое́мъ. а҃ Да восхва́лѧтъ и҆́мѧ є҆гѡ̀ въ ли́цѣ, въ  тѷмпа́нѣ и҆ ѱалти́ри пою́тъ є҆мꙋ̀. в҃ Ꙗ҆́кѡ  бл҃говоли́тъ гдⷭ҇ь въ лю́дехъ   свои́хъ, и҆ вознесе́тъ крѡ́ткїѧ во сп҃се́нїе. а҃  Восхва́лѧтсѧ прпⷣбнїи во сла́вѣ, и҆ возра́дꙋютсѧ на  ло́жахъ свои́хъ. в҃ Возношє́нїѧ бж҃їѧ въ горта́ни и҆́хъ,  и҆ мечѝ ѻ҆бою́дꙋ ѻ҆стры̀ въ рꙋка́хъ и҆́хъ: а҃  Сотвори́ти ѿмще́нїе во ꙗ҆зы́цѣхъ, ѡ҆бличє́нїѧ въ  лю́дехъ. в҃ Свѧза́ти царѝ и҆́хъ пꙋ́ты, и҆ сла̑вныѧ и҆́хъ  рꙋчны́ми ѡ҆ко́вы желѣ́зными. 

\hKv Ѿ здѣ̀ начина́емъ стїхи̑ры на ѕ҃: 

\hKv а҃ Сотвори́ти въ ни́хъ сꙋ́дъ напи́санъ: сла́ва сїѧ̀  бꙋ́детъ всѣ̑мъ прпⷣбнымъ є҆гѡ̀. в҃ Хвали́те бг҃а во  ст҃ы́хъ є҆гѡ̀, хвали́те є҆го̀ во ᲂу҆тверже́нїи си́лы  є҆гѡ̀. 

\hKv На д҃: г҃ Хвали́те є҆го̀ на си́лахъ є҆гѡ̀, хвали́те  є҆го̀ по мно́жествꙋ вели́чествїѧ є҆гѡ̀. д҃ Хвали́те  є҆го̀ во гла́сѣ трꙋ́бнѣмъ, хвали́те є҆го̀ во ѱалти́ри  и҆ гꙋ́слехъ. є҃ Хвали́те є҆го̀ въ тѷмпа́нѣ и҆ ли́цѣ,  хвали́те є҆го̀ во стрꙋ́нахъ и҆ ѻ҆рга́нѣ. ѕ҃ Хвали́те  є҆го̀ въ кѷмва́лѣхъ доброгла́сныхъ,  хвали́те є҆го̀ въ кѷмва́лѣхъ восклица́нїѧ: всѧ́кое  дыха́нїе да хва́литъ гдⷭ҇а. 

\hKv А҆́ще є҆́сть недѣ́лѧ, по стїхи́рахъ: 

\hKv Сла́ва, стїхи́ра є҆ѵⷢ҇льскаѧ. 

\hKv И҆ ны́нѣ, настоѧ́щїй бг҃оро́диченъ: 

\hKv Пребл҃гослове́нна є҆сѝ бцⷣе дв҃о, вопло́щьшимъ бо сѧ  и҆з̾ тебѐ а҆́дъ плѣни́сѧ, а҆да́мъ воззва́сѧ, клѧ́тва  потреби́сѧ, є҆́ѵа свободи́сѧ, сме́рть ᲂу҆мертви́сѧ, и҆ мы̀  ѡ҆жи́хомъ. тѣ́мъ воспѣва́юще вопїе́мъ: бл҃гослове́нъ  хрⷭ҇то́съ бг҃ъ бл҃говоли́вый та́кѡ, сла́ва тебѣ̀. 

\hKv Сла́ва тебѣ̀ показа́вшемꙋ на́мъ свѣ́тъ. 

\hKv Сла́ва въ вы́шнихъ бг҃ꙋ, и҆ на землѝ ми́ръ, въ  человѣ́цѣхъ бл҃говоле́нїе. Хва́лимъ тѧ̀, бл҃гослови́мъ  тѧ̀, кла́нѧемъ ти сѧ, славосло́вимъ тѧ̀, бл҃годари́мъ тѧ̀,  вели́кїѧ ра́ди сла́вы твоеѧ̀. Гдⷭ҇и цр҃ю̀ нбⷭ҇ный, бж҃е  ѻ҆́ч҃е вседержи́телю. Гдⷭ҇и сн҃е є҆диноро́дный і҆и҃се  хрⷭ҇тѐ, и҆ ст҃ы́й дш҃е. Гдⷭ҇и бж҃е, а҆́гнче бж҃їй, сн҃е  ѻ҆ч҃ь, взе́млѧй грѣ́хъ мі́ра, поми́лꙋй на́съ. Взе́млѧй  грѣхѝ мі́ра прїимѝ мл҃твꙋ  на́шꙋ. Сѣдѧ́й ѡ҆деснꙋ́ю ѻ҆ц҃а̀ поми́лꙋй на́съ. Ꙗ҆́кѡ ты̀  є҆сѝ є҆ди́нъ ст҃ъ, ты̀ є҆сѝ є҆ди́нъ гдⷭ҇ь, і҆и҃съ  хрⷭ҇то́съ, въ сла́вꙋ бг҃а ѻ҆ц҃а̀, а҆ми́нь. 

\hKv На всѧ́къ де́нь бл҃гословлю́ тѧ, и҆ восхвалю̀ и҆́мѧ твоѐ во  вѣ́ки, и҆ въ вѣ́къ вѣ́ка. 

\hKv Сподо́би гдⷭ҇и въ де́нь се́й без̾ грѣха̀ сохрани́тисѧ  на́мъ. Бл҃гослове́нъ є҆сѝ гдⷭ҇и бж҃е ѻ҆тє́цъ на́шихъ, и҆  хва́льно, и҆ просла́влено и҆́мѧ твоѐ во вѣ́ки, а҆ми́нь.  Бꙋ́ди гдⷭ҇и млⷭ҇ть твоѧ̀ на на́съ, ꙗ҆́коже ᲂу҆пова́хомъ на  тѧ̀. 

\hKv Бл҃гослове́нъ є҆сѝ гдⷭ҇и, наꙋчи́ мѧ ѡ҆правда́нїємъ  твои́мъ. Три́жды. 

\hKv Гдⷭ҇и, прибѣ́жище бы́лъ є҆сѝ на́мъ въ ро́дъ и҆ ро́дъ.  а҆́зъ рѣ́хъ: гдⷭ҇и, поми́лꙋй мѧ̀, и҆сцѣлѝ дꙋ́шꙋ мою̀,  ꙗ҆́кѡ согрѣши́хъ тебѣ̀. Гдⷭ҇и, къ тебѣ̀ прибѣго́хъ,  наꙋчи́ мѧ твори́ти во́лю твою̀, ꙗ҆́кѡ ты̀ є҆сѝ бг҃ъ мо́й:  ꙗ҆́кѡ ᲂу҆ тебѐ и҆сто́чникъ живота̀, во свѣ́тѣ твое́мъ  ᲂу҆́зримъ свѣ́тъ. Проба́ви млⷭ҇ть твою̀ вѣ́дꙋщымъ тѧ̀.   

\hKv Ст҃ы́й бж҃е, ст҃ы́й крѣ́пкїй, ст҃ы́й безсме́ртный, поми́лꙋй  на́съ. Три́жды. 

\hKv Сла́ва и҆ ны́нѣ: Ст҃ы́й безсме́ртный, поми́лꙋй на́съ. 

\hKv Та́же вы́шшимъ гла́сомъ: Ст҃ы́й бж҃е, ст҃ы́й крѣ́пкїй,  ст҃ы́й безсме́ртный, поми́лꙋй на́съ.  
